\documentclass{llncs}

\usepackage[utf8]{inputenc}
\usepackage[english]{babel}
\usepackage[T1]{fontenc}
\usepackage{graphicx}
\usepackage{booktabs}
\usepackage{amsmath}
\usepackage{url}
\usepackage{lmodern}
\usepackage{nonfloat}

\begin{document}

\title{Automatic Authorship Attribution for\\Croatian Texts}
\author{Tomislav Reicher \and Ivan Krišto \and Igor Belša \and Artur Šilić}
\institute{Faculty of Electrical and Computing Engineering\\
University of Zagreb\\
Unska 3, 10000 Zagreb, Croatia\\
\email{\{tomislav.reicher, ivan.kristo, igor.belsa, artur.silic\}@fer.hr}}

\maketitle

\begin{abstract}
In this work we investigate the system specific for Croatian language. Most
reported methods are language specific, therefore we have developed methods
specific for Croatian language which range from simple, based on stylistic
measures and functional words, to complex which require syntactic analysis of the
Croatian language. We have found that various combinations of methods are very
successful in solving the authorship attribution problem.
% TODO: Obavezno proširiti!! Zasad imamo 60 riječi, a minimum je 70!

\vspace{10pt}
\textbf{Key words:} text classification, SVM, functional words, MSD tagging.
\end{abstract}

\section{Introduction}
Automatic authorship attribution is a task in the field of text classification
that deals with the identification of the author of a given text. It can
be interpreted as a problem of text classification based on linguistic features
specific to certain authors. Problems similar to authorship attribution are
detection of author's age, region or gender \cite{luyckx2005shallow}. The main
concern in computer-based authorship attribution is defining an appropriate
characterization of the text. Such characterization should capture the writing
style of the authors \cite{coyotl2006authorship}.

Authorship attribution can help in document indexing, document filtering and
hierarchical categorization of web pages \cite{luyckx2005shallow}. These
applications are important tasks in the field of information retrieval. 
It is important to say that authorship attribution differs from plagiarism
detection. Plagiarism detection attempts to detect similarities between two substantially different pieces of
work. However, it is unable to determine if they were produced by the same author or not
\cite{de2001mining}.

The problem of authorship attribution can be divided into three categories
\cite{zhao2005effective}: binary, multi-class and single-class (or one-class)
classification. Binary classification solves the problem when the data set
contains the texts written by one of two authors. Multi-class classification is a
generalization of the binary classification when there are more than two authors
in the data set. One-class classification is applied when only some of the texts
from the data set are written by a particular author while the authorship of all
the other texts is unspecified. This classification ascertains whether a given
text belongs to a single known author or not.

This paper presents a study of multi-class classification for the texts written
in Croatian languagem that is oriented on the combination and evaluation of
different text representations features. The rest of the paper is organized as
follows. Section 2 discusses related work on authorship attribution and similar
problems. Section 3 describes the used data set. Section 4 introduces
different types of text representation features we have used. Section 5 describes
classification and Section 6 presents evaluation methods and experiment results. Conclusion and
future work are given in Section 7.

\section{Related Work}
There are several approaches to author attribution in respect of different text
representation features used for the classification. Based on those features
following taxonomy, which is mainly focused on the computational requirements,
can be made \cite{stamatatos2009survey}: \emph{character features},
\emph{lexical features}, \emph{syntactic features}, \emph{semantic features} and
\emph{application--specific features}. Following paragraphs describe these
features in more depth and relate our work with the existing research.

\emph{Character features} are the simplest text representation features, which
consider text as mere sequence of characters, and therby usable for any natural
language and corpus. In that notion various meassures can be defiend such as
characters frequencies, digit frequencies, uppercase and lowercase character
frequencies, punctuation marks frequencies, etc.~\cite{de2001mining}. Another
character based features, which have been proven to be quite successful
\cite{peng2003language,stamatatos2006ensemble}, consider extracting
frequencies of character \emph{n}--grams to be used as features. 

Text representation using \emph{lexical features} is characterized by dividing
the text into the sequence of tokens (words) that group into sentences. Features
straightforwardly derived from that representation are the length of words, the
length of sentences and the richness of vocabulary, used in
\cite{mendenhall1887,holmes1994authorship} . Results achived demonstrate that
these features are not sufficient for the task mainly because of their
significant dependence on text genre and text length. Taking advantage of
features based on frequencies of different words, especially function words,
can produce fairly better results
\cite{argamon2005measuring,uzuner2005comparative,koppel2003exploiting,zhao2005effective}.
Analogusly to character \emph{n}--grams, word \emph{n}--gram features
can be defined for which is shown to be quite successfull too
\cite{keselj2003n,coyotl2006authorship}.

Usage of \emph{syntactic features} is governed by the idea that authors tend to
use similar syntactic patterns unconsciously. Information related to structure of
the language is obtained by an in--depth syntactic analysis of the text and a
single text is characterized by the presence and frequency of certain syntactic
structures. More detailed description of the authorship attribution process based
on syntactic elements can be found in a very influential work by Stamatos et al.
\cite{stamatatos2001computer}.
% Što je s ovim? - \cite{luyckx2005shallow,uzuner2005comparative}

A similar comparison of feature types is shown in
\cite{uzuner2005comparative}. The functional words and syntactic
elements are compared in order to identify the author of a text. It is
concluded that the syntactic elements of expressions are as useful as functional 
words in solving the problem.

Kukushkina et al.\ \cite{kukushkina2001using} explain a method that uses certain
algorithms for data compression in order to identify the author of a text. The
appendix of this paper shows evaluation results for authorship attribution with
different compression algorithms.
% The idea behind
% the method is to divide texts by authors, compress every set with selected
% algorithm and write the size of the archive. To classify text of an unknown author,
% text is added to each set and than the same data compression algorithm is applied. The author of
% the texts in archive which records the smallest increase in size is declared
% as the author of the new text.
A review of a similar method can be found in 
\cite{zhao2005effective}. It is noted that the method has obvious omissions, 
and a citation on another work that proves this is given.

Koppel and Schler \cite{koppel2003exploiting} show the use of grammatical
errors and informal styling (e.g., writing sentences in capital letters) as text features used for authorship attribution.
The method is only applicable for unedited texts (blogs, Internet forums,
newsgroups, e-mail messages, etc.).

Our work is based on the composition and evaluation of various
afore--mentioned text representation features. We use different
character, lexical and syntactical features and adapt them for the use with the
Croatian language.

\section{Data Set}
\label{sec:podatci}

We used an online archive of \emph{proofread} articles (journals) from a daily
Croatian newspaper ``Jutarnji list'', available at
\url{http://www.jutarnji.hr/komentari/}. The data set consists of 4571 texts
written by 25 different authors where the texts are not evenly distributed
among authors. The number of texts per author in the used data set is shown on
Figure \ref{fig:articlesPerAuthor}. The articles are not all the same size
either --- the lowest average number of words in the text per author is 315 words, and
the highest average is 1347 words. An average number of words in text per author for
this data set is 717 words. As one can see, data set we used is very
heterogeneous.

Since the topics in these articles tend to be time--specific, to avoid the
overfitting, we split the set by dates --- 20\% of the newest articles of each author are
taken for testing (hold--out method). Therefore, training data set contains 3425 texts
and testing data set 1146 texts.

\begin{minipage}{0.8\linewidth}
\vspace{10pt}
\centerline{\resizebox{0.7\linewidth}{!}{% GNUPLOT: LaTeX picture
\setlength{\unitlength}{0.240900pt}
\ifx\plotpoint\undefined\newsavebox{\plotpoint}\fi
\sbox{\plotpoint}{\rule[-0.200pt]{0.400pt}{0.400pt}}%
\begin{picture}(1500,900)(0,0)
\sbox{\plotpoint}{\rule[-0.200pt]{0.400pt}{0.400pt}}%
\put(260.0,131.0){\rule[-0.200pt]{4.818pt}{0.400pt}}
\put(240,131){\makebox(0,0)[r]{$10^{1}$}}
\put(1439.0,131.0){\rule[-0.200pt]{4.818pt}{0.400pt}}
\put(260.0,204.0){\rule[-0.200pt]{2.409pt}{0.400pt}}
\put(1449.0,204.0){\rule[-0.200pt]{2.409pt}{0.400pt}}
\put(260.0,247.0){\rule[-0.200pt]{2.409pt}{0.400pt}}
\put(1449.0,247.0){\rule[-0.200pt]{2.409pt}{0.400pt}}
\put(260.0,277.0){\rule[-0.200pt]{2.409pt}{0.400pt}}
\put(1449.0,277.0){\rule[-0.200pt]{2.409pt}{0.400pt}}
\put(260.0,301.0){\rule[-0.200pt]{2.409pt}{0.400pt}}
\put(1449.0,301.0){\rule[-0.200pt]{2.409pt}{0.400pt}}
\put(260.0,320.0){\rule[-0.200pt]{2.409pt}{0.400pt}}
\put(1449.0,320.0){\rule[-0.200pt]{2.409pt}{0.400pt}}
\put(260.0,336.0){\rule[-0.200pt]{2.409pt}{0.400pt}}
\put(1449.0,336.0){\rule[-0.200pt]{2.409pt}{0.400pt}}
\put(260.0,350.0){\rule[-0.200pt]{2.409pt}{0.400pt}}
\put(1449.0,350.0){\rule[-0.200pt]{2.409pt}{0.400pt}}
\put(260.0,363.0){\rule[-0.200pt]{2.409pt}{0.400pt}}
\put(1449.0,363.0){\rule[-0.200pt]{2.409pt}{0.400pt}}
\put(260.0,374.0){\rule[-0.200pt]{4.818pt}{0.400pt}}
\put(240,374){\makebox(0,0)[r]{$10^{2}$}}
\put(1439.0,374.0){\rule[-0.200pt]{4.818pt}{0.400pt}}
\put(260.0,447.0){\rule[-0.200pt]{2.409pt}{0.400pt}}
\put(1449.0,447.0){\rule[-0.200pt]{2.409pt}{0.400pt}}
\put(260.0,490.0){\rule[-0.200pt]{2.409pt}{0.400pt}}
\put(1449.0,490.0){\rule[-0.200pt]{2.409pt}{0.400pt}}
\put(260.0,520.0){\rule[-0.200pt]{2.409pt}{0.400pt}}
\put(1449.0,520.0){\rule[-0.200pt]{2.409pt}{0.400pt}}
\put(260.0,544.0){\rule[-0.200pt]{2.409pt}{0.400pt}}
\put(1449.0,544.0){\rule[-0.200pt]{2.409pt}{0.400pt}}
\put(260.0,563.0){\rule[-0.200pt]{2.409pt}{0.400pt}}
\put(1449.0,563.0){\rule[-0.200pt]{2.409pt}{0.400pt}}
\put(260.0,579.0){\rule[-0.200pt]{2.409pt}{0.400pt}}
\put(1449.0,579.0){\rule[-0.200pt]{2.409pt}{0.400pt}}
\put(260.0,593.0){\rule[-0.200pt]{2.409pt}{0.400pt}}
\put(1449.0,593.0){\rule[-0.200pt]{2.409pt}{0.400pt}}
\put(260.0,606.0){\rule[-0.200pt]{2.409pt}{0.400pt}}
\put(1449.0,606.0){\rule[-0.200pt]{2.409pt}{0.400pt}}
\put(260.0,617.0){\rule[-0.200pt]{4.818pt}{0.400pt}}
\put(240,617){\makebox(0,0)[r]{$10^{3}$}}
\put(1439.0,617.0){\rule[-0.200pt]{4.818pt}{0.400pt}}
\put(260.0,690.0){\rule[-0.200pt]{2.409pt}{0.400pt}}
\put(1449.0,690.0){\rule[-0.200pt]{2.409pt}{0.400pt}}
\put(260.0,733.0){\rule[-0.200pt]{2.409pt}{0.400pt}}
\put(1449.0,733.0){\rule[-0.200pt]{2.409pt}{0.400pt}}
\put(260.0,763.0){\rule[-0.200pt]{2.409pt}{0.400pt}}
\put(1449.0,763.0){\rule[-0.200pt]{2.409pt}{0.400pt}}
\put(260.0,787.0){\rule[-0.200pt]{2.409pt}{0.400pt}}
\put(1449.0,787.0){\rule[-0.200pt]{2.409pt}{0.400pt}}
\put(260.0,806.0){\rule[-0.200pt]{2.409pt}{0.400pt}}
\put(1449.0,806.0){\rule[-0.200pt]{2.409pt}{0.400pt}}
\put(260.0,822.0){\rule[-0.200pt]{2.409pt}{0.400pt}}
\put(1449.0,822.0){\rule[-0.200pt]{2.409pt}{0.400pt}}
\put(260.0,836.0){\rule[-0.200pt]{2.409pt}{0.400pt}}
\put(1449.0,836.0){\rule[-0.200pt]{2.409pt}{0.400pt}}
\put(260.0,849.0){\rule[-0.200pt]{2.409pt}{0.400pt}}
\put(1449.0,849.0){\rule[-0.200pt]{2.409pt}{0.400pt}}
\put(260.0,860.0){\rule[-0.200pt]{4.818pt}{0.400pt}}
\put(240,860){\makebox(0,0)[r]{$10^{4}$}}
\put(1439.0,860.0){\rule[-0.200pt]{4.818pt}{0.400pt}}
\put(260.0,131.0){\rule[-0.200pt]{0.400pt}{4.818pt}}
\put(260,90){\makebox(0,0){ 0}}
\put(260.0,840.0){\rule[-0.200pt]{0.400pt}{4.818pt}}
\put(491.0,131.0){\rule[-0.200pt]{0.400pt}{4.818pt}}
\put(491,90){\makebox(0,0){ 5}}
\put(491.0,840.0){\rule[-0.200pt]{0.400pt}{4.818pt}}
\put(721.0,131.0){\rule[-0.200pt]{0.400pt}{4.818pt}}
\put(721,90){\makebox(0,0){ 10}}
\put(721.0,840.0){\rule[-0.200pt]{0.400pt}{4.818pt}}
\put(952.0,131.0){\rule[-0.200pt]{0.400pt}{4.818pt}}
\put(952,90){\makebox(0,0){ 15}}
\put(952.0,840.0){\rule[-0.200pt]{0.400pt}{4.818pt}}
\put(1182.0,131.0){\rule[-0.200pt]{0.400pt}{4.818pt}}
\put(1182,90){\makebox(0,0){ 20}}
\put(1182.0,840.0){\rule[-0.200pt]{0.400pt}{4.818pt}}
\put(1413.0,131.0){\rule[-0.200pt]{0.400pt}{4.818pt}}
\put(1413,90){\makebox(0,0){ 25}}
\put(1413.0,840.0){\rule[-0.200pt]{0.400pt}{4.818pt}}
\put(260.0,131.0){\rule[-0.200pt]{0.400pt}{175.616pt}}
\put(260.0,131.0){\rule[-0.200pt]{288.839pt}{0.400pt}}
\put(1459.0,131.0){\rule[-0.200pt]{0.400pt}{175.616pt}}
\put(260.0,860.0){\rule[-0.200pt]{288.839pt}{0.400pt}}
\put(859,29){\makebox(0,0){Author}}
\put(306.0,131.0){\rule[-0.200pt]{0.400pt}{93.951pt}}
\put(352.0,131.0){\rule[-0.200pt]{0.400pt}{69.138pt}}
\put(398.0,131.0){\rule[-0.200pt]{0.400pt}{58.057pt}}
\put(444.0,131.0){\rule[-0.200pt]{0.400pt}{47.698pt}}
\put(491.0,131.0){\rule[-0.200pt]{0.400pt}{36.376pt}}
\put(537.0,131.0){\rule[-0.200pt]{0.400pt}{89.374pt}}
\put(583.0,131.0){\rule[-0.200pt]{0.400pt}{30.353pt}}
\put(629.0,131.0){\rule[-0.200pt]{0.400pt}{63.598pt}}
\put(675.0,131.0){\rule[-0.200pt]{0.400pt}{61.911pt}}
\put(721.0,131.0){\rule[-0.200pt]{0.400pt}{21.199pt}}
\put(767.0,131.0){\rule[-0.200pt]{0.400pt}{33.244pt}}
\put(813.0,131.0){\rule[-0.200pt]{0.400pt}{38.303pt}}
\put(859.0,131.0){\rule[-0.200pt]{0.400pt}{37.580pt}}
\put(906.0,131.0){\rule[-0.200pt]{0.400pt}{45.530pt}}
\put(952.0,131.0){\rule[-0.200pt]{0.400pt}{70.102pt}}
\put(998.0,131.0){\rule[-0.200pt]{0.400pt}{54.202pt}}
\put(1044.0,131.0){\rule[-0.200pt]{0.400pt}{64.561pt}}
\put(1090.0,131.0){\rule[-0.200pt]{0.400pt}{64.802pt}}
\put(1136.0,131.0){\rule[-0.200pt]{0.400pt}{8.672pt}}
\put(1182.0,131.0){\rule[-0.200pt]{0.400pt}{82.147pt}}
\put(1228.0,131.0){\rule[-0.200pt]{0.400pt}{124.786pt}}
\put(1275.0,131.0){\rule[-0.200pt]{0.400pt}{72.511pt}}
\put(1321.0,131.0){\rule[-0.200pt]{0.400pt}{77.329pt}}
\put(1367.0,131.0){\rule[-0.200pt]{0.400pt}{71.065pt}}
\put(1413.0,131.0){\rule[-0.200pt]{0.400pt}{88.892pt}}
\put(306,521){\circle{12}}
\put(352,418){\circle{12}}
\put(398,372){\circle{12}}
\put(444,329){\circle{12}}
\put(491,282){\circle{12}}
\put(537,502){\circle{12}}
\put(583,257){\circle{12}}
\put(629,395){\circle{12}}
\put(675,388){\circle{12}}
\put(721,219){\circle{12}}
\put(767,269){\circle{12}}
\put(813,290){\circle{12}}
\put(859,287){\circle{12}}
\put(906,320){\circle{12}}
\put(952,422){\circle{12}}
\put(998,356){\circle{12}}
\put(1044,399){\circle{12}}
\put(1090,400){\circle{12}}
\put(1136,167){\circle{12}}
\put(1182,472){\circle{12}}
\put(1228,649){\circle{12}}
\put(1275,432){\circle{12}}
\put(1321,452){\circle{12}}
\put(1367,426){\circle{12}}
\put(1413,500){\circle{12}}
\put(260.0,131.0){\rule[-0.200pt]{0.400pt}{175.616pt}}
\put(260.0,131.0){\rule[-0.200pt]{288.839pt}{0.400pt}}
\put(1459.0,131.0){\rule[-0.200pt]{0.400pt}{175.616pt}}
\put(260.0,860.0){\rule[-0.200pt]{288.839pt}{0.400pt}}
\end{picture}
}}%
\figcaption{Number of texts per author.}%
\label{fig:articlesPerAuthor}
\end{minipage}

\section{Text Representation}
When constructing an authorship attribution system, the central issuse is the
selection of sufficiently discriminative features. A feature is discriminative if
it is common for one author and rare for all the others. Due to the large number
of authors some complex features are very useful if their distribution is
specific to each author. Moreover, as the texts from dataset greatly differ in
length and topic, it is neccessary to use the features independent of such
variations. If the features were not independent of such variation that
would most certanly reduce generality of system's application and could lead
to a decrease of accurancy (e.g., relating author to concrete topic or terms).

In the following subsections we will describe different features used.

\subsection{Function Words Frequencies ($\mathcal{F}$)}
\label{sec:funkcijske-rijeci}
Function words, such as adverbs, prepositions, conjunctions, or interjections,
are words that have little or no semantic content of their own. They usually
indicate a grammatical relationship or a generic property
\cite{zhao2005effective}. Although one would assume that frequencies of some of
the less used function words would be useful indicator of authors style even the
frequencies of more common function words can adequately distinguish between the
authors. Due to the high frequency of the function words and their significant
roles in the grammar, the author usually has no conscious control over their
usage in a particular text \cite{argamon2005measuring}. They are also
topic--independent. Therefore, function words are good indicators of the author's
style.

It is difficult to predict whether these words will give equally good results for
different languages. Moreover, despite the abundance of research in this field,
due to various languages, types and sizes of the texts, it is currently
impossible to conclude if these features are generally effective
\cite{zhao2005effective}.

In addition to function words in this work we also consider frequencies of
auxiliary verbs and pronouns as their frequencies might also be
representative of the style of different authors. This makes the set of totaly
652 function words used.

Label $\mathcal{F}$ denotes evaluation result of this feature in Table
\ref{tbl:eval}.

\subsection{Idf Weighted Function Words Frequency ($\mathcal{I}$)}
\label{sec:funkcijske-rijeci-idf}

Usage of features based on function words frequencies often implies the
problem with the quantifying how important, for the sake of discrimination, a
specific given function word is \cite{diederich2003authorship}.

To cope with that problem we used a combination of $L_p$ normalization of the
length and transformation of the function word occurrence frequency, in
particular \emph{idf} (inverse document frequency) measure.

Idf measure is defined as \cite{diederich2003authorship}
\begin{equation}
F_{idf}(t_k) = \log \frac{n_d}{n_d(t_k)},
\label{equ:idf}
\end{equation}
where $n_d(t_k)$ is the number of texts from learning data set that contain word
$t_k$ and $n_d$ the total number of the texts in that learning data set. The
shown measure gives high values for words that appear in a small number of
texts and are thus very discriminatory.

As the \emph{idf} measure uses only the information of the pressence of a
certain function word in the texts ignoring frequency of that word, a word that appears
many times in one single text and once in all the others gets the same value as
the one which appears once in all of the texts. Therefore it is necessary to
multiply the obtained \emph{idf} measure, of the given function word, with the
occurrence frequency of that word in the observed text.

\subsection{Lexical Categories Frequency ($\mathcal{C}$)}
\label{sec:rijeci-grupe}
Next three features we used are syntax--based features and thus some sort of NLP
(Natural Language Processing) tool was required. Croatian language is
morphologically complex and it is problematic to construct a robust NLP tool.
With the tool we used, given in \cite{snajder08automatic}, we were able to get
% TODO: Ovdje je neki upitnik bio.. o čemu se radi? - ``text. ?? The method''
morphosyntactic descriptions of each word in the text. The method does not
use context information and therefore cannot distinguish between different
homographs --- words with same spelling but with different meaning and probably
different POS too --- so all possible POS tags for a given word are considered.

Simplest syntax--based feature we used are features based on lexical category
frequency, similar to one used in \cite{kukushkina2001using}. The given text is
tagged with the use of NLP tool and thus lexical category for each word in the text is determinated. Features are obtained by
counting the number of occurrence of different lexical categories, which are
then normalized by the total number of words in the text. If for some word more
then one lexical category is obtained, all different categories where counted. 
The used categories are adverbs, adpositions, conjunctions, particles,
interjections, nouns, verbs, adjectives and pronouns. In addition, category
``unknown'' was introduced for all the words whose lexical category was not
determinated (names, places, etc.).

\subsection{Word Morphosyntactical Categories ($\mathcal{M}$)}
\label{sec:morphosyntactic}

More complex syntax--based features we used take advantage of morphosyntactic
description of a word. Each word can be described by the set of different
morphosyntactic as are features that take The feature vector is created by
counting the appearances of morphologic categories for every word in a text, and dividing them by the number of words in
the text. Counted morphologic categories are \emph{case}, \emph{degree},
\emph{form}, \emph{gender}, \emph{number} and \emph{person}.

\subsection{Word Part--Of--Speech \emph{n}-grams Frequency ($\mathcal{N}_1$ \&
$\mathcal{N}_2$)}
\label{sec:ngrami-tipova}
The two proposed features are based on word part--of--speech \emph{n}-grams
frequency. Word parts--of--speech and their morphosyntactic descriptors are obtained by POS
(Part--Of--Speech) and MSD (morphosyntactic) tagging for Croatian language
\cite{snajder08automatic}. Having the corresponding part--of--speech for
each word in the text makes the usage \emph{n}-grams as features
possible. As \emph{n}-gram features can produce very large dimensionality, 
only 3-grams are considered. In addition, POS tagging used is not perfect. 

%FIXME: pos patterns, not n-grams
The first feature proposed uses the words parts--of--speech
(nouns, verbs, adjectives, pronouns, conjunctions, interjections and
prepositions) to form various \emph{n}-grams and count their frequencies. For
example, word 3-gram ``Adam i Eva'' (``Adam and Eve'') forms ``noun
conjunction noun'' trigram. The second feature proposed uses only words
parts--of--speech information for nouns, verbs and adjectives and for other word
parts--of--speech it uses words as they are, therefore ``Adam i Eva'' transforms
to ``noun i noun''. Due to many different pronouns, conjunctions, interjections
and prepositions that make many different \emph{n}-grams, frequency filtering
is applied --- only the frequency of 500 most frequent 3-grams in the
training data set is considered. Used dimension reduction method is not optimal,
therefore in the future work other methods should be evaluated, such as information
gain, $\chi^2$ test, mutual information, maximum relevance, minimum redundancy
or classification with sparse SVM, logistic regression or na\"ive Bayes.

\subsection{Other Features}
\label{sec:znacajke-manje}

Other features we used are simple features: punctation marks, vowels, words
length and sentence length frequencies.

A set of following punctuation marks is used: ``.'', ``,'', ``!'', ``?'',
``''', ``"'', ``-'', ``:'', ``;'', ``+'', ``*''. Their appearance in the text is counted and
the result is normalized by the total number of characters in the text. 

Features based on the frequency of vowel occurence (a, e, i, o, u) are
obtained in an equal manner.

The frequencies of words lengths are obtained by counting the lengths of all
the words form a text and then normalizing them by the total number of words in
text. To enforce consistency, the words longer than 10 characters are counted as
if they were 10 characters long.

The sentence length frequency is obtained in a similar procedure.
For the same reason as with the words length, sentences longer than 20 words are
counted as if they were 20 words long.

All features suggested here int this subsection have weak discriminatory power
on their own. However, they proved very useful in the combination with other features, as
shown in Section \ref{sec:evaluacija}.

\section{Classification}
% Representation of documents by real number vectors  enables easy use of
% classifiers that search decision functions, i.e., boundaries in vector
% space.
All the features mentioned in this work use some sort of frequency information
that makes it possible to represent them by real valued feature vectors. Having
features in a form of vectors, for the classification, we used an SVM (Support
Vector Machine) with radial basis function as the kernel. It is shown that, with
the use of parameter selection, linear SVM is a special case of an SVM with RBF
kernel \cite{keerthi2003asymptotic}, which removes the need to consider a linear
SVM as a potential classifier.

Before the learning process and classification with the SVM, we scale the data,
real valued feature vectors, to ensure equal contribution of every attribute to
the classification. Components of every feature vector are scaled to an interval
$[0, 1]$.
%  according to following expression:
% \begin{equation}
% x^{s}_{i,j} = \frac{x_{i,j} - \min_{i}\; x_{i,j}}{\max_{i}\; x_{i,j}
% - \min_{i}\; x_{i,j}}
% \end{equation}
% where $x^{s}_{i,j}$ is scaled component $j$ of vector $\mathbf{x_i}$,
% $\min_{i}\; x_{i,j}$ is minimum value of attribute $j$ among all vectors
% $\mathbf{x_i}$ and $\max_{i}\; x_{i,j}$ is maximum value of attribute $j$ among
% all vectors. If we denote the resulting minimum and maximum values as follows:
% \begin{eqnarray}
% M_i & = \max_{i}\; x_{i,j} \\
% m_i & = \min_{i}\; x_{i,j}
% \end{eqnarray}
% then, the unknown vector $\mathbf{x}$ before classification is scaled as:
% \begin{equation}
% x^{s}_{j} = \frac{x_j-m_i}{M_i-m_i}
% \end{equation}

% TODO: REPHRASE
The searching of the appropriate parameters $(C, \gamma)$ is done by the means
of cross-validation: using the 5--fold cross--validation on the
learning set parameters $(C, \gamma)$ that give the highest accuracy are
selected and the SVM classifier is learned using them. 
The accuracy of classification is measured by the expression:
\begin{equation}
acc = \frac{n_c}{N}, % ako bude problema, ovo staviti kao ``inline'' jednadžbu.
\end{equation}
where $n_c$ is the number of correctly classified texts and $N$ is the total number of
texts.
Parameters $(C, \gamma)$ that were considered are: $C \in \{2^{-5}, 2^{-4},
\cdots , 2^{15}\}$, $\gamma \in \{2^{-15}, 2^{-14}, \cdots, 2^3\}$ \cite{CC01a}.

\begin{table*}
\begin{center}%
\caption{Evaluation of Different Features}%
\begin{tabular}{c c c r@{.}l}%
\toprule%
Features & Accuracy [\%] & $C$ & \multicolumn{2}{c}{$\gamma$} \\
\midrule
$\mathcal{F}$ & 88.39 & 8192 & 0 & 125\\
$\mathcal{I}$ & 87.96 & 8192 & 0 & 125\\
$\mathcal{C}$ & 44.50 & 512 & 2 & 0\\
$\mathcal{P}$ & 57.50 & 8192 & 0 & 125\\
$\mathcal{V}$ & 30.54 & 128 & 0 & 125\\
$\mathcal{L}$ & 43.19 & 128 & 0 & 125\\
$\mathcal{S}$ & 42.32 & 128 & 0 & 125\\
$\mathcal{N}_1$ & 71.29 & 512 & 0 & 125\\
$\mathcal{N}_2$ & 76.09 & 512 & 0 & 125\\
$\mathcal{M}$ & 61.17 & 512 & 0 & 125\\
$\mathcal{C}$, $\mathcal{M}$ & 63.17 & 8192 & 0 & 03125\\
$\mathcal{P}$, $\mathcal{F}$ & 91.71 & 8 & 0 & 03125\\
$\mathcal{F}$, $\mathcal{M}$ & 91.18 & 128 & 0 & 03125\\
$\mathcal{F}$, $\mathcal{N}_1$ & 89.44 & 128 & 0 & 03125\\
$\mathcal{F}$, $\mathcal{N}_2$ & 88.48 & 128 & 0 & 03125\\
$\mathcal{I}$, $\mathcal{M}$ & 90.84 & 128 & 0 & 03125\\
$\mathcal{N}_1$, $\mathcal{M}$ & 71.38 & 128 & 0 & 03125\\
$\mathcal{I}$, $\mathcal{M}$, $\mathcal{C}$ & 91.36 & 128 & 0 & 03125\\
$\mathcal{P}$, $\mathcal{F}$, $\mathcal{L}$ & \textbf{93.11} & 128 & 0 & 03125\\
$\mathcal{P}$, $\mathcal{F}$, $\mathcal{L}$, $\mathcal{M}$ & \textbf{93.46} & 32768 & 0 & 03125\\
$\mathcal{F}$, $\mathcal{M}$, $\mathcal{C}$, $\mathcal{N}_1$ & 89.62 & 128 & 0 & 03125\\
$\mathcal{S}$, $\mathcal{P}$, $\mathcal{F}$, $\mathcal{V}$, $\mathcal{L}$, $\mathcal{M}$ & 93.19 & 128 & 0 & 03125\\
$\mathcal{S}$, $\mathcal{P}$, $\mathcal{F}$, $\mathcal{V}$, $\mathcal{L}$, $\mathcal{M}$, $\mathcal{N}_1$ & 92.41 & 128 & 0 & 03125\\
\bottomrule%
\end{tabular}%
\label{tbl:eval}%
\end{center}
\end{table*}

\section{Evaluation}
\label{sec:evaluacija}
Classification success is measured by the ratio of correctly classified texts and
the total number of texts in the training set (accuracy). First we tested all the
features separately and then we tested those features in different combinations.
The process of SVM parameter selection is very time consuming so not all feature
combinations were tested. Results of the evalutaion are shown in Table
\ref{tbl:eval}. Columns ``$C$'' and ``$\gamma$'' denote parameters of SVM
classifier optimal for the selected features.



% Highest accuracy during cross--validation is achived by combination of
% methods marked as $\mathcal{S}$, $\mathcal{P}$, $\mathcal{F}$, $\mathcal{V}$,
% $\mathcal{L}$ and $\mathcal{M}$.

%% TODO: Što s ovim?
% Total accuracy doesn't explains behaviour of classifier for every class by
% itself. For every class precision and recall are calculated and by them we
% calculate total weighted $F$ measure. For particular class $c$, precision is
% ratio of number of correctly classified documents in $c$ with number of all
% documents which are classified as $c$. Recall is ratio of number of correctly
% classified documents in $c$ with number of all documents in $c$. $F$ measure is
% calculated for every class $c_i$ according to following expression:
% \begin{equation}
% F_i = \frac{2 \cdot precision_i \cdot recall_i}{precision_i + recall_i}
% \end{equation}
% where $precision_i$ and $recall_i$ are measures of precision and recall for
% class $c_i$.
% 
% Total weighted measure is calculated by following expression:
% \begin{equation}
% F_u = \frac{\sum^{n}_i |c_i|\cdot F_i}{\sum^n_i|c_i|}
% \end{equation}
% where $n$ is total number of classes, $|c_i|$ number of documents in class
% $c_i$ and $F_i$ a $F$ measure for class $c_i$.
% 
% Weighted $F$ measure at our data set for testing gives value of 87\%.

\section{Conclusion}
It is shown that the authorship attribution problem, in morphologically complex
language such as the Croatian, can be successfully solved using some relatively
simple features. Our results with 93\% accuracy are quite notable considering
that the data set used was very heterogeneous with texts of rather small
size (on average less than 1000 words). The resulst also fit very well in the
interval of previous reported results, which range from 70\% to 97\% \cite{coyotl2006authorship,keselj2003n,luyckx2005shallow,stamatatos2001computer}.

The comparision of the results has proved difficult due to the different types of
data sets (e.g., poems, newspaper articles, e-mails) and the different types of
problems (binary, multi-class or single-class classifications). There are no
relevant data sets for comparision \cite{zhao2005effective}. A significant impact
on the complexity of this problem is the total number of authors considered and
the variety of document samples.

In the future work, methods based on word and character \emph{n}-grams suggested
in \cite{keselj2003n,peng2003language,coyotl2006authorship} should be
evaluated. Moreover, evaluation has to be performed on different types of data sets
such as poems, newspaper articles or book data sets.

\section*{Acknowledgement}
% Hvala dragom Bogu i drugu Titu što umijem da pročitam u potpisu tko je autor teksta!
This research has been supported by the Croatian Ministry of Science, Education
and Sports under the grant No.036-1300646-1986.

\bibliographystyle{splncs}
\bibliography{literatura}

\end{document}
