\documentclass{article}
\usepackage[utf8]{inputenc}
\usepackage[croatian]{babel}
\usepackage[T1]{fontenc}
\usepackage{lmodern}
\usepackage{algorithmic}
\usepackage{algorithm}
\usepackage{longtable}
\usepackage{graphicx}
\usepackage{booktabs}
%\usepackage{hyperref}
% Da bi se promjenio stil citiranja umjesto:
% [authoryear, round]
% staviti:
% [numbers, square]
\usepackage[authoryear, round]{natbib}
\usepackage{amsmath}
\usepackage{subfig}
\usepackage{fixltx2e}
\usepackage{url}
\usepackage{textcomp}
\usepackage{float}

\newcommand{\engl}[1]{(engl.~\emph{#1})}

\begin{document}

\title{Automatizirano prepoznavanje autora teksta\\Projekt iz Strojnog učenja}
\author{Igor Belša \and Tomislav Reicher \and Ivan Krišto}

\maketitle

\thispagestyle{empty}

\tableofcontents

\newpage

\section{Uvod}
Problem automatiziranog prepoznavanja autora teksta \engl{Authorship Attribution}
može se promatrati kao problem klasifikacije teksta na temelju njegovih
lingvističkih značajki. Problemi slični prepoznavanju autora su prepoznavanje
dobi autora, regije iz koje autor potječe ili spola \citep{luyckx2005shallow}.
Glavni problem računalno potpomognutog prepoznavanja autora teksta je određivanje
prikladnog skupa karakteristika dokumenta koje opisuju stil autora
\citep{coyotl2006authorship}.

\citet{kukushkina2001using} navode da se problem prepoznavanja autora teksta
obrađuje još od 1915.~godine u radu ``\emph{Izv.~otd.~russkogo jazyka i
slovesnosti, Imp.~akad.~nauk.}'', Morozov, N., A.~kojem se već 1916.~pridružio
utemeljitelj teorije Markovljevih lanaca, Markov, A., A.,~radom ``\emph{On some
application of statistical method}''.

Prepoznavanje autora može pomoći pri indeksiranju dokumenata, filtriranju ili
hijerarhiskoj kategorizaciji web stranica i web pretraživača
\citep{luyckx2005shallow}. Bitno je napomenuti kako se prepoznavanje autora i
njegovih karakteristika razlikuje od detekcije plagijata. Detekcija
plagijata pokušava odrediti sličnost između dva u suštini različita
djela, ali nije u mogućnosti otkriti jesu li proizvod istog autora
\citep{de2001mining}.

Problem prepoznavanja autora može se podijeliti u tri kategorije
\citep{zhao2005effective}: binarnu, višerazrednu i jednorazrednu klasifikaciju.
Binarna klasifikacija rješava problem kad je za skup tekstova poznato da
pripadaju jednom od dva autora. Višerazredna klasifikacija je proširenje
binarne na više autora. Jednorazredna se primjenjuje u situacijama kad je
dio tekstova napisan od poznatog autora, a autori ostalih tekstova su
nepoznati. Ta klasifikacija pokušava odgovoriti na pitanje pripada li neki
tekst poznatom autoru ili ne.

U ovom radu obrađen je problem višerazredne klasifikacije za tekstove na
hrvatskom jeziku. Naglasak je stavljen na evaluiranje različitih skupova
karakteristika.

Rad je organiziran tako da su u 2.~odjeljku opisane metode koje su u drugim
radovima korištene za problem prepoznavanja autora i srodne probleme. Odjeljak 3
daje kratki pregled sustava, način na koji su komponente u sustavu povezane i na
koji način se sustav može koristiti. Slijedi 4.~odjeljak s opisom korištenih
podataka. Najbitniji za rad je odijeljak 5. U njemu je dan opis korištenih
metoda za izvlačenje značajki iz tekstova, popis svih korištenih značajki i
njihove specifičnosti te evaluacija. Na 5.~odjeljak usko se veže 6.~sa
opisom klasifikacije. Implementacija je opisana u 7., a zaključak je
dan u 8. odjeljku.

\newpage

\section{Srodna rješenja}
\citet{coyotl2006authorship} dijele postojeće metode prepoznavanja autora
teksta na tri glavna pristupa:
\begin{description}
\item[Uzimanje mjera stila kao značajki dokumenta:] Za pristup su
karakteristične značajke koje u obzir uzimaju duljinu riječi i rečenica te
bogatstvo vokabulara. Navodi se da takve značajke nisu dovoljne za rješavanje
problema, da ovise o žanru teksta te da gube smisao ako se primjenjuju nad
kratkim tekstovima.
\item[Korištenje sintaksnih obilježja kao značajki:] Korištenje oznaka stila
autora \engl{style markers} kao značajki -- informacije vezane uz strukturu
jezika koje su skupljene dubinskom analizom sintaksne analize dokumenta. Tekst se
karakterizira pojavom i frekvencijom pojedinih sintaksnih struktura. Navedena
karakterizacija je detaljna i relevantna. Nažalost, računalno je zahtjevna i
zahtjeva alate za obradu teksta (npr.~POS označivači \engl{Part-Of-Speech taggers},
sintaksni parseri i sl.). Očit je i utjecaj duljine teksta na rezultat klasifikacije.
\item[Značajke dokumenta temeljene na riječima:] Pristup se grana na tri
različite metode. \emph{Prva} karakterizira dokumente koristeći skup funkcijskih
riječi\footnote{Riječi koje nemaju sadržajno značenje.} (njihovu pojavu i
frekvenciju pojavljivanja), ignorirajući sadržajne riječi \engl{content words}
jer su one usko vezane uz temu dokumenta. Metoda pokazuje dobre rezultate, ali na
nju utječe veličina dokumenta. \emph{Druga} grana dokument promatra kao hrpu
riječi \engl{bag-of-words} te koristi jedno-sadržajne riječi \engl{single
content words} kao značajke dokumenta. Metoda je vrlo robusna i daje odlične rezultate
kad postoji primjetna korelacija između autora i teme o kojoj pišu. \emph{Treća}
grana razmatra $n$-grame riječi kao značajke, npr.~značajke koje se sastoje od
sekvenci $n$ uzastopnih riječi. Radi prebrzog rasta značajki koriste se samo
$n$-grami do tri riječi.
\end{description}

\citet{kukushkina2001using} navode postupak koji koristi algoritme za kompresiju
podataka u svrhu prepoznavanja autora. U dodatku svog rada navode eksperimentalne
rezultate prepoznavanja autora sa različitim kompresijskim algoritmima. Ideja
postupka je dokumente za učenje podijeliti u skupove po autorima i svaki skup
sažeti odabranim algoritmom te zabilježiti veličine arhiva. Tekst nepoznatog
autora se dodaje svakom skupu za učenje te se primjenjuje isti algoritam
kompresije. Arhiva koja bilježi najmanji porast veličine proglašava se
pripadnom arhivom novog teksta, tj.~autor njenog sadržaja, autor teksta. Osvrt
na sličan postupak spominje se u radu \citep{zhao2005effective}. Taj rad navodi
da postupak ima očite propuste te navodi rad koji to potvrđuje.

\citet{koppel2003exploiting} opisuju korištenje gramatičkih greški i
neformalnog stiliziranja (npr.~pisanje rečenica velikim slovima u svrhu
naglašavanja i sl.) kao značajki dokumenta u svrhu prepoznavanja autora. Metoda
je primjenjiva samo na neuređivane tekstove (npr.~blogovi, Internet forumi,
\emph{news grupe}, e-mail poruke, i sl.).

\citet{peng2003language} navode mogućnost izrade jezičnog modela pomoću $n$-grama te
njegovog korištenja za prepoznavanje autora. Sličan postupak je opisan u
\citep{coyotl2006authorship}.

Uspješnost funkcijskih riječi nad kolokacijama\footnote{Dvije riječi koje se
pojavljuju na određenoj udaljenosti jedna od druge.} navodi se u
\citep{argamon2005measuring} -- iako, radi redukcije informacija pri uzimanju
funkcijskih riječi kao značajki, intuitivno bi se moglo zaključiti suprotno,
odnosno pogrešno. Također se navodi sumnja da zapravo diskriminatorna osobina
kolokacija dolazi od frekventnih riječi koje one sadržavaju. Uz to, navedeno je da znatno veći
broj dokumenata za učenje od broja značajki izuzetno dobro smanjuje
vjerojatnost pretreniranosti klasifikatora. Slično je navedeno i u
radu \citep{banko2001scaling} koji obrađuje značaj veličine korpusa pri
općenitoj obradi prirodnog jezika.

Slična usporedba tipova značajki je napravljena u \citep{uzuner2005comparative}
pri čemu su se uspoređivale funkcijske riječi i sintaksni elementi dokumenata u
svrhu prepoznavanja autora. Došli su do zaključka da sintaksni elementi izraza
nisu korisni koliko funkcijske riječi pri rješavanju navedenog problema.

Opis postupka prepoznavanja autora teksta pomoću sintaksnih elemenata jezika
tj.~alata za obradu prirodnog jezika može se naći u izuzetno utjecajnom radu
\citep{stamatatos2001computer}. Razne vrste oznaka stila \engl{style markers}
također navode \citet{diri2003automatic} te \citet{luyckx2005shallow}.

Osim opisa navedenog postupka, \citet{stamatatos2001computer}, kao često
korištenu značajku pri prepoznavanju autora navode raznolikost riječnika autora.
Ona se mjeri omjerom unikatnih riječi i ukupne veličine teksta ili brojanjem
riječi koje se pojavljuju samo jednom (\emph{hapax legomena}), riječi koje se
pojavljuju samo dva puta (\emph{dis legomena}) i sl. Te mjere su usko vezano uz
dužinu teksta. Rad navodi funkcije date u
\citep{yule1944statistical,honore1979some} koje bi trebale riješiti taj problem,
tj. trebale bi biti konstantne s obzirom na dužinu teksta. Analiza sličnih
funkcija može se naći u radu \citep{tweedie1998variable} koji tvrdi da većina
takvih funkcija nije neovisna o duljini teksta. Zaključuje se da je raznolikost
riječnika nestabilna značajka za tekstove kraće od 1000 riječi.

Problem ovisnosti mjera o dužini teksta radi što uspješnije klasifikacije pomoću
SVM-a\footnote{Support Vector Machines} navode \citet{diederich2003authorship}.
Za izbjegavanje ovisnosti koriste kombinaciju $L_p$ normalizacije duljine i
transformacije frekvencija pojave termina kao što je
\emph{idf}\footnote{Inverse document frequency} mjera.

Idf mjera je definirana sa
\begin{equation}
F_{idf}(t_k) = \log \frac{n_d}{n_d(t_k)},
\label{equ:idf}
\end{equation}
pri čemu je $n_d(t_k)$ broj dokumenata koji sadržavaju termin $t_k$ te $n_d$
ukupan broj dokumenata. Navedena mjera daje visoke vrijednosti za dokumente koji
se pojavljuju u malom broju dokumenata te je time jako diskriminatorna. 

% Redundancija je mjera asimetričnosti razdiobe te se definira sa
% \begin{equation}
% R(t_k) = \log n_d + \sum_{i=1}^{n_d} \frac{f(t_k, d_i)}{f(t_k)} \cdot \log \frac{f(t_k, d_i)}{f(t_k)},
% \label{equ:redundancija}
% \end{equation}
% pri čemu je $f(t_k, d_i)$ frekvencija pojavljivanja termina $t_k$ u dokumentu
% $d_i$, $f(t_k)$ frekvencija pojavljivanja termina u svim dokumentima te $n_d$
% ukupan broj dokumenata. $R(t_k)$ je mjera koliko razdioba termina $t_k$ u raznim
% dokumentima odstupa od uniformne razdiobe.

\section{Model sustava}
Namjena razvijenog sustava je prepoznati autora sustavu dosad neviđenog teksta.
Autor se može prepoznati samo ako je sustav naučen nad skupom tekstova koji je
sadržavao i tekstove tog autora. Sustav ima dva načina korištenja, učenje i
određivanje autora. Da bi se sustav mogao koristiti za određivanje autora, prvo
mora biti naučen. Apstraktni model dan je slikom \ref{fig:model-sustava}, a
detalji o načinu korištenja se nalaze u \ref{sec:implementacija}.~odjeljku.

\begin{figure}[htb]
\begin{center}		
\begin{picture}(470,90)		
\put(5,70){\makebox(40,15){\texttt{XML}}}
\put(38,77){\vector(1,0){15}}
\put(55,70){\framebox(90,15){\textit{Izlučivanje značajki}}}		
\put(145,77){\vector(1,0){15}}
\put(160,70){\framebox(90,15){\textit{Učenje}}}		
\put(250,77){\vector(1,0){15}}
\put(285,70){\makebox(40,15){\texttt{prepoznavatelj}}}

\put(1,30){\makebox(40,15){\texttt{tekst}}}
\put(38,37){\vector(1,0){15}}
\put(55,30){\framebox(90,15){\textit{Izlučivanje značajki}}}		
\put(145,37){\vector(1,0){15}}
\put(160,30){\framebox(90,15){\textit{Prepoznavanje}}}		
\put(250,37){\vector(1,0){15}}
\put(205,15){\vector(0,1){15}}
\put(165,0){\makebox(80,15){\texttt{naučeni prepoznavatelj}}}
\put(262,30){\makebox(40,15){\texttt{autor}}}
\end{picture}		
\caption{Model sustava}		
\label{fig:model-sustava}
\end{center}		
\end{figure}

\section{Korišteni podatci}
\label{sec:podatci}
Pri izradi sustava korištena je mrežna arhiva kolumni ``Jutarnjeg lista'' do
14.~studenog, 2009.~-- ``Komentari i kolumne na aktualna događanja u
Hrvatskoj - Jutarnji.hr''.\footnote{\url{http://www.jutarnji.hr/komentari/}}

Arhiva je pohranjena kao XML\footnote{Extensible Markup Language --
\url{http://www.w3.org/XML/}} dokument po KTLab\footnote{Knowledge Technologies
Lab, FER -- \url{http://ktlab.fer.hr}} \texttt{documentSet} shemi dostupnoj na
\url{http://ktlab.fer.hr/download/documentSet.xsd}. Budući da se veliki broj
članaka u mrežnoj arhivi našao u više kopija, one su pri izradi XML arhive
preskočene.

U arhivi se nalaze tekstovi 25 autora sa ukupno 4359 tekstova koji zajedno broje
2998766 riječi. Detaljne statistike arhive nalaze se na slikama
\ref{fig:articlesPerAuthor}\footnote{Broj tekstova za autora na mjestu 21 nije
greška. Radi se o Živku Kustiću koji je imao ukupno 1285 objavljenih kolumni do
datuma kada je napravljena arhiva.}, \ref{fig:wordsPerAuthor} i
\ref{fig:avgWordsPerAuthorArticle}.

Da bi se izbjeglo učenje teme, arhiva je podijeljena po razdobljima, tj.~za
testiranje je odvojeno $20\%$ najnovijih tekstova svakog autora.

\begin{figure}[htb]
\begin{center}
% GNUPLOT: LaTeX picture
\setlength{\unitlength}{0.240900pt}
\ifx\plotpoint\undefined\newsavebox{\plotpoint}\fi
\sbox{\plotpoint}{\rule[-0.200pt]{0.400pt}{0.400pt}}%
\begin{picture}(1500,900)(0,0)
\sbox{\plotpoint}{\rule[-0.200pt]{0.400pt}{0.400pt}}%
\put(260.0,131.0){\rule[-0.200pt]{4.818pt}{0.400pt}}
\put(240,131){\makebox(0,0)[r]{$10^{1}$}}
\put(1439.0,131.0){\rule[-0.200pt]{4.818pt}{0.400pt}}
\put(260.0,204.0){\rule[-0.200pt]{2.409pt}{0.400pt}}
\put(1449.0,204.0){\rule[-0.200pt]{2.409pt}{0.400pt}}
\put(260.0,247.0){\rule[-0.200pt]{2.409pt}{0.400pt}}
\put(1449.0,247.0){\rule[-0.200pt]{2.409pt}{0.400pt}}
\put(260.0,277.0){\rule[-0.200pt]{2.409pt}{0.400pt}}
\put(1449.0,277.0){\rule[-0.200pt]{2.409pt}{0.400pt}}
\put(260.0,301.0){\rule[-0.200pt]{2.409pt}{0.400pt}}
\put(1449.0,301.0){\rule[-0.200pt]{2.409pt}{0.400pt}}
\put(260.0,320.0){\rule[-0.200pt]{2.409pt}{0.400pt}}
\put(1449.0,320.0){\rule[-0.200pt]{2.409pt}{0.400pt}}
\put(260.0,336.0){\rule[-0.200pt]{2.409pt}{0.400pt}}
\put(1449.0,336.0){\rule[-0.200pt]{2.409pt}{0.400pt}}
\put(260.0,350.0){\rule[-0.200pt]{2.409pt}{0.400pt}}
\put(1449.0,350.0){\rule[-0.200pt]{2.409pt}{0.400pt}}
\put(260.0,363.0){\rule[-0.200pt]{2.409pt}{0.400pt}}
\put(1449.0,363.0){\rule[-0.200pt]{2.409pt}{0.400pt}}
\put(260.0,374.0){\rule[-0.200pt]{4.818pt}{0.400pt}}
\put(240,374){\makebox(0,0)[r]{$10^{2}$}}
\put(1439.0,374.0){\rule[-0.200pt]{4.818pt}{0.400pt}}
\put(260.0,447.0){\rule[-0.200pt]{2.409pt}{0.400pt}}
\put(1449.0,447.0){\rule[-0.200pt]{2.409pt}{0.400pt}}
\put(260.0,490.0){\rule[-0.200pt]{2.409pt}{0.400pt}}
\put(1449.0,490.0){\rule[-0.200pt]{2.409pt}{0.400pt}}
\put(260.0,520.0){\rule[-0.200pt]{2.409pt}{0.400pt}}
\put(1449.0,520.0){\rule[-0.200pt]{2.409pt}{0.400pt}}
\put(260.0,544.0){\rule[-0.200pt]{2.409pt}{0.400pt}}
\put(1449.0,544.0){\rule[-0.200pt]{2.409pt}{0.400pt}}
\put(260.0,563.0){\rule[-0.200pt]{2.409pt}{0.400pt}}
\put(1449.0,563.0){\rule[-0.200pt]{2.409pt}{0.400pt}}
\put(260.0,579.0){\rule[-0.200pt]{2.409pt}{0.400pt}}
\put(1449.0,579.0){\rule[-0.200pt]{2.409pt}{0.400pt}}
\put(260.0,593.0){\rule[-0.200pt]{2.409pt}{0.400pt}}
\put(1449.0,593.0){\rule[-0.200pt]{2.409pt}{0.400pt}}
\put(260.0,606.0){\rule[-0.200pt]{2.409pt}{0.400pt}}
\put(1449.0,606.0){\rule[-0.200pt]{2.409pt}{0.400pt}}
\put(260.0,617.0){\rule[-0.200pt]{4.818pt}{0.400pt}}
\put(240,617){\makebox(0,0)[r]{$10^{3}$}}
\put(1439.0,617.0){\rule[-0.200pt]{4.818pt}{0.400pt}}
\put(260.0,690.0){\rule[-0.200pt]{2.409pt}{0.400pt}}
\put(1449.0,690.0){\rule[-0.200pt]{2.409pt}{0.400pt}}
\put(260.0,733.0){\rule[-0.200pt]{2.409pt}{0.400pt}}
\put(1449.0,733.0){\rule[-0.200pt]{2.409pt}{0.400pt}}
\put(260.0,763.0){\rule[-0.200pt]{2.409pt}{0.400pt}}
\put(1449.0,763.0){\rule[-0.200pt]{2.409pt}{0.400pt}}
\put(260.0,787.0){\rule[-0.200pt]{2.409pt}{0.400pt}}
\put(1449.0,787.0){\rule[-0.200pt]{2.409pt}{0.400pt}}
\put(260.0,806.0){\rule[-0.200pt]{2.409pt}{0.400pt}}
\put(1449.0,806.0){\rule[-0.200pt]{2.409pt}{0.400pt}}
\put(260.0,822.0){\rule[-0.200pt]{2.409pt}{0.400pt}}
\put(1449.0,822.0){\rule[-0.200pt]{2.409pt}{0.400pt}}
\put(260.0,836.0){\rule[-0.200pt]{2.409pt}{0.400pt}}
\put(1449.0,836.0){\rule[-0.200pt]{2.409pt}{0.400pt}}
\put(260.0,849.0){\rule[-0.200pt]{2.409pt}{0.400pt}}
\put(1449.0,849.0){\rule[-0.200pt]{2.409pt}{0.400pt}}
\put(260.0,860.0){\rule[-0.200pt]{4.818pt}{0.400pt}}
\put(240,860){\makebox(0,0)[r]{$10^{4}$}}
\put(1439.0,860.0){\rule[-0.200pt]{4.818pt}{0.400pt}}
\put(260.0,131.0){\rule[-0.200pt]{0.400pt}{4.818pt}}
\put(260,90){\makebox(0,0){ 0}}
\put(260.0,840.0){\rule[-0.200pt]{0.400pt}{4.818pt}}
\put(491.0,131.0){\rule[-0.200pt]{0.400pt}{4.818pt}}
\put(491,90){\makebox(0,0){ 5}}
\put(491.0,840.0){\rule[-0.200pt]{0.400pt}{4.818pt}}
\put(721.0,131.0){\rule[-0.200pt]{0.400pt}{4.818pt}}
\put(721,90){\makebox(0,0){ 10}}
\put(721.0,840.0){\rule[-0.200pt]{0.400pt}{4.818pt}}
\put(952.0,131.0){\rule[-0.200pt]{0.400pt}{4.818pt}}
\put(952,90){\makebox(0,0){ 15}}
\put(952.0,840.0){\rule[-0.200pt]{0.400pt}{4.818pt}}
\put(1182.0,131.0){\rule[-0.200pt]{0.400pt}{4.818pt}}
\put(1182,90){\makebox(0,0){ 20}}
\put(1182.0,840.0){\rule[-0.200pt]{0.400pt}{4.818pt}}
\put(1413.0,131.0){\rule[-0.200pt]{0.400pt}{4.818pt}}
\put(1413,90){\makebox(0,0){ 25}}
\put(1413.0,840.0){\rule[-0.200pt]{0.400pt}{4.818pt}}
\put(260.0,131.0){\rule[-0.200pt]{0.400pt}{175.616pt}}
\put(260.0,131.0){\rule[-0.200pt]{288.839pt}{0.400pt}}
\put(1459.0,131.0){\rule[-0.200pt]{0.400pt}{175.616pt}}
\put(260.0,860.0){\rule[-0.200pt]{288.839pt}{0.400pt}}
\put(859,29){\makebox(0,0){Author}}
\put(306.0,131.0){\rule[-0.200pt]{0.400pt}{93.951pt}}
\put(352.0,131.0){\rule[-0.200pt]{0.400pt}{69.138pt}}
\put(398.0,131.0){\rule[-0.200pt]{0.400pt}{58.057pt}}
\put(444.0,131.0){\rule[-0.200pt]{0.400pt}{47.698pt}}
\put(491.0,131.0){\rule[-0.200pt]{0.400pt}{36.376pt}}
\put(537.0,131.0){\rule[-0.200pt]{0.400pt}{89.374pt}}
\put(583.0,131.0){\rule[-0.200pt]{0.400pt}{30.353pt}}
\put(629.0,131.0){\rule[-0.200pt]{0.400pt}{63.598pt}}
\put(675.0,131.0){\rule[-0.200pt]{0.400pt}{61.911pt}}
\put(721.0,131.0){\rule[-0.200pt]{0.400pt}{21.199pt}}
\put(767.0,131.0){\rule[-0.200pt]{0.400pt}{33.244pt}}
\put(813.0,131.0){\rule[-0.200pt]{0.400pt}{38.303pt}}
\put(859.0,131.0){\rule[-0.200pt]{0.400pt}{37.580pt}}
\put(906.0,131.0){\rule[-0.200pt]{0.400pt}{45.530pt}}
\put(952.0,131.0){\rule[-0.200pt]{0.400pt}{70.102pt}}
\put(998.0,131.0){\rule[-0.200pt]{0.400pt}{54.202pt}}
\put(1044.0,131.0){\rule[-0.200pt]{0.400pt}{64.561pt}}
\put(1090.0,131.0){\rule[-0.200pt]{0.400pt}{64.802pt}}
\put(1136.0,131.0){\rule[-0.200pt]{0.400pt}{8.672pt}}
\put(1182.0,131.0){\rule[-0.200pt]{0.400pt}{82.147pt}}
\put(1228.0,131.0){\rule[-0.200pt]{0.400pt}{124.786pt}}
\put(1275.0,131.0){\rule[-0.200pt]{0.400pt}{72.511pt}}
\put(1321.0,131.0){\rule[-0.200pt]{0.400pt}{77.329pt}}
\put(1367.0,131.0){\rule[-0.200pt]{0.400pt}{71.065pt}}
\put(1413.0,131.0){\rule[-0.200pt]{0.400pt}{88.892pt}}
\put(306,521){\circle{12}}
\put(352,418){\circle{12}}
\put(398,372){\circle{12}}
\put(444,329){\circle{12}}
\put(491,282){\circle{12}}
\put(537,502){\circle{12}}
\put(583,257){\circle{12}}
\put(629,395){\circle{12}}
\put(675,388){\circle{12}}
\put(721,219){\circle{12}}
\put(767,269){\circle{12}}
\put(813,290){\circle{12}}
\put(859,287){\circle{12}}
\put(906,320){\circle{12}}
\put(952,422){\circle{12}}
\put(998,356){\circle{12}}
\put(1044,399){\circle{12}}
\put(1090,400){\circle{12}}
\put(1136,167){\circle{12}}
\put(1182,472){\circle{12}}
\put(1228,649){\circle{12}}
\put(1275,432){\circle{12}}
\put(1321,452){\circle{12}}
\put(1367,426){\circle{12}}
\put(1413,500){\circle{12}}
\put(260.0,131.0){\rule[-0.200pt]{0.400pt}{175.616pt}}
\put(260.0,131.0){\rule[-0.200pt]{288.839pt}{0.400pt}}
\put(1459.0,131.0){\rule[-0.200pt]{0.400pt}{175.616pt}}
\put(260.0,860.0){\rule[-0.200pt]{288.839pt}{0.400pt}}
\end{picture}

\end{center}
\caption{Broj tekstova po autoru.}
\label{fig:articlesPerAuthor}
\end{figure}

\begin{figure}[htb]
\begin{center}
% GNUPLOT: LaTeX picture
\setlength{\unitlength}{0.240900pt}
\ifx\plotpoint\undefined\newsavebox{\plotpoint}\fi
\sbox{\plotpoint}{\rule[-0.200pt]{0.400pt}{0.400pt}}%
\begin{picture}(1500,900)(0,0)
\sbox{\plotpoint}{\rule[-0.200pt]{0.400pt}{0.400pt}}%
\put(260.0,131.0){\rule[-0.200pt]{4.818pt}{0.400pt}}
\put(240,131){\makebox(0,0)[r]{$10^{4}$}}
\put(1439.0,131.0){\rule[-0.200pt]{4.818pt}{0.400pt}}
\put(260.0,241.0){\rule[-0.200pt]{2.409pt}{0.400pt}}
\put(1449.0,241.0){\rule[-0.200pt]{2.409pt}{0.400pt}}
\put(260.0,305.0){\rule[-0.200pt]{2.409pt}{0.400pt}}
\put(1449.0,305.0){\rule[-0.200pt]{2.409pt}{0.400pt}}
\put(260.0,350.0){\rule[-0.200pt]{2.409pt}{0.400pt}}
\put(1449.0,350.0){\rule[-0.200pt]{2.409pt}{0.400pt}}
\put(260.0,386.0){\rule[-0.200pt]{2.409pt}{0.400pt}}
\put(1449.0,386.0){\rule[-0.200pt]{2.409pt}{0.400pt}}
\put(260.0,415.0){\rule[-0.200pt]{2.409pt}{0.400pt}}
\put(1449.0,415.0){\rule[-0.200pt]{2.409pt}{0.400pt}}
\put(260.0,439.0){\rule[-0.200pt]{2.409pt}{0.400pt}}
\put(1449.0,439.0){\rule[-0.200pt]{2.409pt}{0.400pt}}
\put(260.0,460.0){\rule[-0.200pt]{2.409pt}{0.400pt}}
\put(1449.0,460.0){\rule[-0.200pt]{2.409pt}{0.400pt}}
\put(260.0,479.0){\rule[-0.200pt]{2.409pt}{0.400pt}}
\put(1449.0,479.0){\rule[-0.200pt]{2.409pt}{0.400pt}}
\put(260.0,496.0){\rule[-0.200pt]{4.818pt}{0.400pt}}
\put(240,496){\makebox(0,0)[r]{$10^{5}$}}
\put(1439.0,496.0){\rule[-0.200pt]{4.818pt}{0.400pt}}
\put(260.0,605.0){\rule[-0.200pt]{2.409pt}{0.400pt}}
\put(1449.0,605.0){\rule[-0.200pt]{2.409pt}{0.400pt}}
\put(260.0,669.0){\rule[-0.200pt]{2.409pt}{0.400pt}}
\put(1449.0,669.0){\rule[-0.200pt]{2.409pt}{0.400pt}}
\put(260.0,715.0){\rule[-0.200pt]{2.409pt}{0.400pt}}
\put(1449.0,715.0){\rule[-0.200pt]{2.409pt}{0.400pt}}
\put(260.0,750.0){\rule[-0.200pt]{2.409pt}{0.400pt}}
\put(1449.0,750.0){\rule[-0.200pt]{2.409pt}{0.400pt}}
\put(260.0,779.0){\rule[-0.200pt]{2.409pt}{0.400pt}}
\put(1449.0,779.0){\rule[-0.200pt]{2.409pt}{0.400pt}}
\put(260.0,804.0){\rule[-0.200pt]{2.409pt}{0.400pt}}
\put(1449.0,804.0){\rule[-0.200pt]{2.409pt}{0.400pt}}
\put(260.0,825.0){\rule[-0.200pt]{2.409pt}{0.400pt}}
\put(1449.0,825.0){\rule[-0.200pt]{2.409pt}{0.400pt}}
\put(260.0,843.0){\rule[-0.200pt]{2.409pt}{0.400pt}}
\put(1449.0,843.0){\rule[-0.200pt]{2.409pt}{0.400pt}}
\put(260.0,860.0){\rule[-0.200pt]{4.818pt}{0.400pt}}
\put(240,860){\makebox(0,0)[r]{$10^{6}$}}
\put(1439.0,860.0){\rule[-0.200pt]{4.818pt}{0.400pt}}
\put(260.0,131.0){\rule[-0.200pt]{0.400pt}{4.818pt}}
\put(260,90){\makebox(0,0){ 0}}
\put(260.0,840.0){\rule[-0.200pt]{0.400pt}{4.818pt}}
\put(491.0,131.0){\rule[-0.200pt]{0.400pt}{4.818pt}}
\put(491,90){\makebox(0,0){ 5}}
\put(491.0,840.0){\rule[-0.200pt]{0.400pt}{4.818pt}}
\put(721.0,131.0){\rule[-0.200pt]{0.400pt}{4.818pt}}
\put(721,90){\makebox(0,0){ 10}}
\put(721.0,840.0){\rule[-0.200pt]{0.400pt}{4.818pt}}
\put(952.0,131.0){\rule[-0.200pt]{0.400pt}{4.818pt}}
\put(952,90){\makebox(0,0){ 15}}
\put(952.0,840.0){\rule[-0.200pt]{0.400pt}{4.818pt}}
\put(1182.0,131.0){\rule[-0.200pt]{0.400pt}{4.818pt}}
\put(1182,90){\makebox(0,0){ 20}}
\put(1182.0,840.0){\rule[-0.200pt]{0.400pt}{4.818pt}}
\put(1413.0,131.0){\rule[-0.200pt]{0.400pt}{4.818pt}}
\put(1413,90){\makebox(0,0){ 25}}
\put(1413.0,840.0){\rule[-0.200pt]{0.400pt}{4.818pt}}
\put(260.0,131.0){\rule[-0.200pt]{0.400pt}{175.616pt}}
\put(260.0,131.0){\rule[-0.200pt]{288.839pt}{0.400pt}}
\put(1459.0,131.0){\rule[-0.200pt]{0.400pt}{175.616pt}}
\put(260.0,860.0){\rule[-0.200pt]{288.839pt}{0.400pt}}
\put(859,29){\makebox(0,0){Autor}}
\put(306.0,131.0){\rule[-0.200pt]{0.400pt}{152.008pt}}
\put(352.0,131.0){\rule[-0.200pt]{0.400pt}{117.800pt}}
\put(398.0,131.0){\rule[-0.200pt]{0.400pt}{59.261pt}}
\put(444.0,131.0){\rule[-0.200pt]{0.400pt}{46.494pt}}
\put(491.0,131.0){\rule[-0.200pt]{0.400pt}{60.466pt}}
\put(537.0,131.0){\rule[-0.200pt]{0.400pt}{104.791pt}}
\put(583.0,131.0){\rule[-0.200pt]{0.400pt}{27.463pt}}
\put(629.0,131.0){\rule[-0.200pt]{0.400pt}{93.469pt}}
\put(675.0,131.0){\rule[-0.200pt]{0.400pt}{87.928pt}}
\put(721.0,131.0){\rule[-0.200pt]{0.400pt}{24.572pt}}
\put(767.0,131.0){\rule[-0.200pt]{0.400pt}{28.185pt}}
\put(813.0,131.0){\rule[-0.200pt]{0.400pt}{34.208pt}}
\put(859.0,131.0){\rule[-0.200pt]{0.400pt}{17.827pt}}
\put(906.0,131.0){\rule[-0.200pt]{0.400pt}{42.880pt}}
\put(952.0,131.0){\rule[-0.200pt]{0.400pt}{90.819pt}}
\put(998.0,131.0){\rule[-0.200pt]{0.400pt}{78.533pt}}
\put(1044.0,131.0){\rule[-0.200pt]{0.400pt}{91.301pt}}
\put(1090.0,131.0){\rule[-0.200pt]{0.400pt}{62.393pt}}
\put(1136.0,131.0){\rule[-0.200pt]{0.400pt}{9.395pt}}
\put(1182.0,131.0){\rule[-0.200pt]{0.400pt}{127.195pt}}
\put(1228.0,131.0){\rule[-0.200pt]{0.400pt}{146.467pt}}
\put(1275.0,131.0){\rule[-0.200pt]{0.400pt}{100.455pt}}
\put(1321.0,131.0){\rule[-0.200pt]{0.400pt}{119.968pt}}
\put(1367.0,131.0){\rule[-0.200pt]{0.400pt}{79.015pt}}
\put(1413.0,131.0){\rule[-0.200pt]{0.400pt}{145.263pt}}
\put(306,762){\circle{12}}
\put(352,620){\circle{12}}
\put(398,377){\circle{12}}
\put(444,324){\circle{12}}
\put(491,382){\circle{12}}
\put(537,566){\circle{12}}
\put(583,245){\circle{12}}
\put(629,519){\circle{12}}
\put(675,496){\circle{12}}
\put(721,233){\circle{12}}
\put(767,248){\circle{12}}
\put(813,273){\circle{12}}
\put(859,205){\circle{12}}
\put(906,309){\circle{12}}
\put(952,508){\circle{12}}
\put(998,457){\circle{12}}
\put(1044,510){\circle{12}}
\put(1090,390){\circle{12}}
\put(1136,170){\circle{12}}
\put(1182,659){\circle{12}}
\put(1228,739){\circle{12}}
\put(1275,548){\circle{12}}
\put(1321,629){\circle{12}}
\put(1367,459){\circle{12}}
\put(1413,734){\circle{12}}
\put(260.0,131.0){\rule[-0.200pt]{0.400pt}{175.616pt}}
\put(260.0,131.0){\rule[-0.200pt]{288.839pt}{0.400pt}}
\put(1459.0,131.0){\rule[-0.200pt]{0.400pt}{175.616pt}}
\put(260.0,860.0){\rule[-0.200pt]{288.839pt}{0.400pt}}
\end{picture}

\end{center}
\caption{Broj riječi po autoru.}
\label{fig:wordsPerAuthor}
\end{figure}

\begin{figure}[htb]
\begin{center}
% GNUPLOT: LaTeX picture
\setlength{\unitlength}{0.240900pt}
\ifx\plotpoint\undefined\newsavebox{\plotpoint}\fi
\sbox{\plotpoint}{\rule[-0.200pt]{0.400pt}{0.400pt}}%
\begin{picture}(1500,900)(0,0)
\sbox{\plotpoint}{\rule[-0.200pt]{0.400pt}{0.400pt}}%
\put(200.0,131.0){\rule[-0.200pt]{4.818pt}{0.400pt}}
\put(180,131){\makebox(0,0)[r]{ 200}}
\put(1439.0,131.0){\rule[-0.200pt]{4.818pt}{0.400pt}}
\put(200.0,253.0){\rule[-0.200pt]{4.818pt}{0.400pt}}
\put(180,253){\makebox(0,0)[r]{ 400}}
\put(1439.0,253.0){\rule[-0.200pt]{4.818pt}{0.400pt}}
\put(200.0,374.0){\rule[-0.200pt]{4.818pt}{0.400pt}}
\put(180,374){\makebox(0,0)[r]{ 600}}
\put(1439.0,374.0){\rule[-0.200pt]{4.818pt}{0.400pt}}
\put(200.0,496.0){\rule[-0.200pt]{4.818pt}{0.400pt}}
\put(180,496){\makebox(0,0)[r]{ 800}}
\put(1439.0,496.0){\rule[-0.200pt]{4.818pt}{0.400pt}}
\put(200.0,617.0){\rule[-0.200pt]{4.818pt}{0.400pt}}
\put(180,617){\makebox(0,0)[r]{ 1000}}
\put(1439.0,617.0){\rule[-0.200pt]{4.818pt}{0.400pt}}
\put(200.0,739.0){\rule[-0.200pt]{4.818pt}{0.400pt}}
\put(180,739){\makebox(0,0)[r]{ 1200}}
\put(1439.0,739.0){\rule[-0.200pt]{4.818pt}{0.400pt}}
\put(200.0,860.0){\rule[-0.200pt]{4.818pt}{0.400pt}}
\put(180,860){\makebox(0,0)[r]{ 1400}}
\put(1439.0,860.0){\rule[-0.200pt]{4.818pt}{0.400pt}}
\put(200.0,131.0){\rule[-0.200pt]{0.400pt}{4.818pt}}
\put(200,90){\makebox(0,0){ 0}}
\put(200.0,840.0){\rule[-0.200pt]{0.400pt}{4.818pt}}
\put(442.0,131.0){\rule[-0.200pt]{0.400pt}{4.818pt}}
\put(442,90){\makebox(0,0){ 5}}
\put(442.0,840.0){\rule[-0.200pt]{0.400pt}{4.818pt}}
\put(684.0,131.0){\rule[-0.200pt]{0.400pt}{4.818pt}}
\put(684,90){\makebox(0,0){ 10}}
\put(684.0,840.0){\rule[-0.200pt]{0.400pt}{4.818pt}}
\put(926.0,131.0){\rule[-0.200pt]{0.400pt}{4.818pt}}
\put(926,90){\makebox(0,0){ 15}}
\put(926.0,840.0){\rule[-0.200pt]{0.400pt}{4.818pt}}
\put(1168.0,131.0){\rule[-0.200pt]{0.400pt}{4.818pt}}
\put(1168,90){\makebox(0,0){ 20}}
\put(1168.0,840.0){\rule[-0.200pt]{0.400pt}{4.818pt}}
\put(1411.0,131.0){\rule[-0.200pt]{0.400pt}{4.818pt}}
\put(1411,90){\makebox(0,0){ 25}}
\put(1411.0,840.0){\rule[-0.200pt]{0.400pt}{4.818pt}}
\put(200.0,131.0){\rule[-0.200pt]{0.400pt}{175.616pt}}
\put(200.0,131.0){\rule[-0.200pt]{303.293pt}{0.400pt}}
\put(1459.0,131.0){\rule[-0.200pt]{0.400pt}{175.616pt}}
\put(200.0,860.0){\rule[-0.200pt]{303.293pt}{0.400pt}}
\put(829,29){\makebox(0,0){Autor}}
\put(248.0,131.0){\rule[-0.200pt]{0.400pt}{147.913pt}}
\put(297.0,131.0){\rule[-0.200pt]{0.400pt}{170.557pt}}
\put(345.0,131.0){\rule[-0.200pt]{0.400pt}{32.762pt}}
\put(394.0,131.0){\rule[-0.200pt]{0.400pt}{38.544pt}}
\put(442.0,131.0){\rule[-0.200pt]{0.400pt}{120.209pt}}
\put(491.0,131.0){\rule[-0.200pt]{0.400pt}{40.953pt}}
\put(539.0,131.0){\rule[-0.200pt]{0.400pt}{52.275pt}}
\put(587.0,131.0){\rule[-0.200pt]{0.400pt}{91.542pt}}
\put(636.0,131.0){\rule[-0.200pt]{0.400pt}{85.760pt}}
\put(684.0,131.0){\rule[-0.200pt]{0.400pt}{79.979pt}}
\put(733.0,131.0){\rule[-0.200pt]{0.400pt}{43.362pt}}
\put(781.0,131.0){\rule[-0.200pt]{0.400pt}{42.157pt}}
\put(829.0,131.0){\rule[-0.200pt]{0.400pt}{17.827pt}}
\put(878.0,131.0){\rule[-0.200pt]{0.400pt}{37.099pt}}
\put(926.0,131.0){\rule[-0.200pt]{0.400pt}{75.643pt}}
\put(975.0,131.0){\rule[-0.200pt]{0.400pt}{91.060pt}}
\put(1023.0,131.0){\rule[-0.200pt]{0.400pt}{87.928pt}}
\put(1072.0,131.0){\rule[-0.200pt]{0.400pt}{22.645pt}}
\put(1120.0,131.0){\rule[-0.200pt]{0.400pt}{86.001pt}}
\put(1168.0,131.0){\rule[-0.200pt]{0.400pt}{121.895pt}}
\put(1217.0,131.0){\rule[-0.200pt]{0.400pt}{17.104pt}}
\put(1265.0,131.0){\rule[-0.200pt]{0.400pt}{79.979pt}}
\put(1314.0,131.0){\rule[-0.200pt]{0.400pt}{122.859pt}}
\put(1362.0,131.0){\rule[-0.200pt]{0.400pt}{34.208pt}}
\put(1411.0,131.0){\rule[-0.200pt]{0.400pt}{150.803pt}}
\put(248,745){\circle{12}}
\put(297,839){\circle{12}}
\put(345,267){\circle{12}}
\put(394,291){\circle{12}}
\put(442,630){\circle{12}}
\put(491,301){\circle{12}}
\put(539,348){\circle{12}}
\put(587,511){\circle{12}}
\put(636,487){\circle{12}}
\put(684,463){\circle{12}}
\put(733,311){\circle{12}}
\put(781,306){\circle{12}}
\put(829,205){\circle{12}}
\put(878,285){\circle{12}}
\put(926,445){\circle{12}}
\put(975,509){\circle{12}}
\put(1023,496){\circle{12}}
\put(1072,225){\circle{12}}
\put(1120,488){\circle{12}}
\put(1168,637){\circle{12}}
\put(1217,202){\circle{12}}
\put(1265,463){\circle{12}}
\put(1314,641){\circle{12}}
\put(1362,273){\circle{12}}
\put(1411,757){\circle{12}}
\put(200.0,131.0){\rule[-0.200pt]{0.400pt}{175.616pt}}
\put(200.0,131.0){\rule[-0.200pt]{303.293pt}{0.400pt}}
\put(1459.0,131.0){\rule[-0.200pt]{0.400pt}{175.616pt}}
\put(200.0,860.0){\rule[-0.200pt]{303.293pt}{0.400pt}}
\end{picture}

\end{center}
\caption{Prosječan broj riječi u tekstu po autoru.}
\label{fig:avgWordsPerAuthorArticle}
\end{figure}

\section{Izvlačenje značajki iz dokumenata i evaluacija}
Glavni problem konstrukcije sustava za prepoznavanje autora je odabir dovoljno
diskriminatornih značajki teksta. Značajka je diskriminatorna ako je česta kod
jednog autora, a rijetka kod drugih. Zbog velikoga broja autora izuzetno su
korisne složene značajke čija je razdioba karakteristična za svakog autora
(npr.~razdioba nekih riječi).

Značajke je moguće kombinirati. Na ovaj način, ako se koriste složene značajke,
značajka može predstavljati više razdiobi i biti jače diskriminatorna.

Prilikom stvaranja vektora značajki potrebno je voditi brigu da značajka ne
ovisi o duljini i sadržaju teksta. Takav utjecaj smanjuje općenitost primjene
sustava i može dovesti do smanjenja uspješnosti (npr.~povezivanje autora s
konkretnom temom).

Značajke se prikazuju u obliku vektora realnih brojeva što omogućava relativno
jednostavnu klasifikaciju. Skup dokumenata se prikazuje kao skup vektora jednakih
dimenzija. Kombiniranje značajki je spajanje (nadovezivanje) vektora značajki.

% Potrebno je niz jezičnih znakova na prikladan način prilagoditi za klasifikaciju, primjerice
% prikazati značajke teksta u obliku vektora realnih brojeva. Nakon ove faze, vektori značajki se dalje
% mogu na standardni način bez problema koristiti u bilo kojem tipu poznatog klasifikatora.

Ovisno o odabranom modelu značajki, analiza teksta i stvaranje vektora značajki
može varirati od računski vrlo trivijalnih operacija pa do vrlo zahtjevnih
algoritama (npr.~značajke dobivene sintaksnom analizom prirodnog jezika).


\subsection{Frekvencije funkcijskih riječi}
Funkcijske riječi su riječi koje promatrane samostalno nemaju semantičko
značenje, poput priloga, prijedloga, veznika, čestica, uzvika ili riječi koje
opisuju količinu. Obično ukazuju na gramatičke odnose ili općenita svojstva
\citep{zhao2005effective}.

Privlačnost mjerenja frekvencije pojavljivanja funkcijskih riječi za
prepoznavanje autora leži u činjenici da su one pokazatelj stila pisanja. Neke
manje poznate funkcijske riječi --- poput prijedloga \emph{onkraj},
\emph{namjesto} ili \emph{zavrh} --- se rijeđe koriste i mogu vrlo dobro
sugerirati autora. Međutim, čak se i frekvencija korištenja češćih funkcijskih
riječi može vrlo dobro iskoristiti za razlikovanje autora. Pokazuje se kako zbog
velike frekvencije korištenja funkcijskih riječi i njihove uloge u gramatici
autor obično nema svjesnu kontrolu nad njihovom uporabom u pojedinom tekstu
\citep{argamon2005measuring}.

Velika prednost funkcijskih riječi je neovisnost o temi pisanja. Nevezano uz
sadržaj, autori prilikom pisanja nesvjesno koriste funkcijske riječi indikativno
njihovo vlastitom stilu. S druge strane, ovisnost o jeziku može biti veliki
nedostatak. Promjenom jezika dokumenata, potrebno je ažurirati i skup funkcijskih
riječi s kojima sustav radi. Teško je predvidjeti kako će se tako promijenjen
sustav ponašati i hoće li funkcijske riječi dati jednako dobre rezultate za
različite jezike. Iako postoji više istraživanja na ovu temu, zbog različitih
jezika, vrsta i veličine dokumenata teško je zaključiti koliko su ovakve metode
općenito učinkovite \citep{zhao2005effective}.

\subsection{Frekvencije grupa tipova funkcijskih riječi}
\subsection{Frekvencije funkcijskih riječi sa IDF težinama}
\subsection{Frekvencije interpunkcijskih znakova}
\subsection{Frekvencije samoglasnika}
\subsection{Frekvencije duljina riječi}

\section{Klasifikacija}
Prikaz dokumenata vektorima realnih brojeva omogućava jednostavno korištenje
klasifikatora koji traže decizijske funkcije, tj.~granice u vektorskom
prostoru.

% Značajke korištene u ovom radu su frekvencije pojavljivanja određenih događaja
% (riječi, znakova) te se grupiranjem frekvencija značajke mogu predočiti pomoću
% vektora u vektorskom prostoru. Zbog navedene činjenice moguće je iskoristiti neki
% od mnogih razvijenih klasifikatora koji traže decizijske funkcije, granice u
% vektorskom prostoru. Jedan od klasifikatora koji traži optimalnu granicu između
% razreda predočenih u prostoru je stroj s potporenim vektorima \engl{Support
% vector machine}. 

U radu je za klasifikaciju korišten stroj s potpornim vektorima \engl{Support
vector machine}. Izvorno, SVM traži optimalnu linearnu granicu, odnosno
hiperravninu, kako bi razdvojio različite razrede predstavljene skupom vektora u
vektorskome prostoru. Iskorištavanjem jezgrenog trika \engl{kernel trick} isti
klasifikator moguće je primijeniti za traženje proizvoljno nelinearne granice
između različitih razreda. Pri tome često korištena jezgrena funkcija je
radijalna bazna funkcija odnosno Gaussova jezgra:
\begin{equation}
k(\mathbf{x_i},\mathbf{x_j})=\exp(-\gamma \|\mathbf{x_i} - \mathbf{x_j}\|^2).
\end{equation}

Pokazano je da uz odabir ispravnih parametara \citep{keerthi2003asymptotic} linearni SVM
predstavlja specijalni slučaj SVM--a s radijalnom baznom funkcijom čime
se isključuje potreba za korištenjem linearnog SVM--a kao potencijalnog
klasifikatora.

Učenjem SVM--a traže se parametri pretpostavljenog oblika decizijske funkcije,
koja će ispravno klasificirati sve uzorke u skupu za učenje, što ponekad zbog
šuma u podacima ili njihove distribucije u prostoru nije moguće. Rješenje tog
problema nalazi se u korištenju SVM klasifikatora s mekim granicama definiranima
parametrom $C$ koji dozvoljava odstupanja od ispravne klasifikacije svih podataka
u skupu za učenje s ciljem bolje generalizacije nad još neviđenim skupom
podataka. Parametri SVM--a koji utječu na moć generalizacije nad neviđenim skupom
za ispitivanje su tako $C$ (parametar meke granice) i $\gamma$ (parametar
radijalne bazne funkcije). Oni zajedno čine prostor parametara čijom je pretragom
potrebno pronaći vrijednosti parametara, tj.~odabrati onaj model, koji će dati
SVM klasifikator s najmanjom pogreškom generalizacije.

Pretraga parametra odvija se odabirom modela s parametrima $(C, \gamma)$ iz skupa
$\left (C = {2^{-5}, 2^{-4}, \ldots , 2^{15}},  \gamma = {2^{-15}, 2^{-14},
\ldots, 2^3} \right )$ \citep{CC01a} koji daju najveću točnost klasifikacije u
procesu cross--validacije nad skupom za učenje. Točnost klasifikacije mjeri se formulom:
\begin{equation}
acc = \frac{n_c}{N},
\end{equation}
pri čemu je $n_c$ broj točno klasificiranih članaka, a $N$ ukupan broj članaka
nad kojima je provedeno testiranje. Nakon što su pronađeni parametri modela $(C,
\gamma)$, koji daju najveću točnost, klasifikator s navedenim parametrima ponovo
se uči na potpunom skupu za učenje. Parametri SVM klasifikatora korišteni u ovom
radu su $C = 16$ i $\gamma = 0.25$.

% \section{Evaluacija}
% % TODO: Napisati!1

\section{Implementacija}
\label{sec:implementacija}
Implementacija je napisana u programskim jezicima Javi (glavni sustav) te
Pythonu (skripta \texttt{JutarnjiKolumneArhiver.py}). Navedena skripta služi za
dohvat arhive tekstova sa ``Komentari i kolumne na aktualna događanja u
Hrvatskoj -- Jutarnji.hr''.\footnote{\url{http://www.jutarnji.hr/komentari/}}

% TODO: UCD Dohvat članaka i izlučivanje značajki
% TODO: UCD Učenje sustava
% TODO: UCD Prepoznavanje autora.

Za klasifikaciju iskorištena je biblioteka \emph{libsvm} \citep{CC01a}. Veza
\emph{libsvm} biblioteke i sustava izvedena je \texttt{LibsvmRecognizer}
razredom koji istovremeno implementira sučelja \texttt{RecognizerTrainer} i
\texttt{AuthorRecognizer}.

Korisnik sustavu može pristupiti iz komandne linije preko \texttt{CLITrainer}
(učenje sustava) i \texttt{CLIRecognizer} (prepoznavatelj autora) ili preko
grafičkog sučelja iz \texttt{GUI} razreda. Svi navedeni razredi nalaze se u
\texttt{hr.fer.zemris.aa.main} paketu. Primjer pokretanja sustava iz
komandne linije:
\begin{verbatim}
  java hr.fer.zemris.aa.main.CLITrainer \
  skup_za_ucenje-skup_aa.xml \
  skup_aa.model
  
  java hr.fer.zemris.aa.main.CLIRecognizer \
  tekst_nepoznatog_autora.txt \
  skup_aa.model
\end{verbatim}


\section{Zaključak}
% TODO: Napisati! Zasad je naveden samo daljni rad..

% TODO dodaj ovo iz komentara:
% Navesti kako su naši rezultati izuzetno dobri u odnosu na prijavljene (one koje
% spominju u člancima), no ne možemo reći da je metoda prvoklasna jer ne postoji
% relevantan skup za usporedbu kao što navode \citep{zhao2005effective}, iako se
% u literaturi često može naći pozivanje na rad \citep{stamatatos2001computer} i
% njihov ``Grčki skup'' (npr.~\citep{keselj2003n}).

U daljnjem radu potrebno je evaluirati metode temeljene na $n$-gramima riječi i
slova poput onih opisanih u
\citep{keselj2003n,peng2003language,coyotl2006authorship} te korištenje oznaka
stila autora (engl. style markers) kao znacajki – informacije vezane uz strukturu
jezika koje su skupljene dubinskom analizom sintaksne analize dokumenta.
\citep{stamatatos2001computer,diri2003automatic,luyckx2005shallow}.

Uz navedeno, potrebno je provesti evaluaciju nad ispitnim korpusima različitih
prosječnih duljina tekstova (npr.~pjesme, novinski članci, knjige).


\bibliography{literatura}
\bibliographystyle{plainnat}

\newpage
\appendix
\section{Doprinos svakog člana tima}
\begin{description}
\item[Ivan Krišto:] dokumentacija, nabavljanje uzoraka za učenje i testiranje,
dizajn sustava, povezivanje libsvma, izrada izlučitelja značajki.
\item[Igor Belša:] dokumentacija, izrada alata za pretvaranje XML dokumenata u
značajke, dizajn sustava, testiranje, izrada izlučitelja značajki.
\item[Tomislav Reicher:] dokumentacija, dizajn sustava, podrška za traženje
parametara klasifikatora, podešavanje izlučitelja značajki, testiranje, izrada
izlučitelja značajki.
\end{description}

\section{Raspodjela bodova}
% TODO: Staviti!
\begin{description}
\item[Ivan Krišto:] p1\%
\item[Igor Belša:] p2\%
\item[Tomislav Reicher:] p3\%
\end{description}

\end{document}
