\documentclass{article}
\usepackage[utf8]{inputenc}
\usepackage[croatian]{babel}
\usepackage[T1]{fontenc}
\usepackage{lmodern}
\usepackage{algorithmic}
\usepackage{algorithm}
\usepackage{longtable}
\usepackage{graphicx}
\usepackage{booktabs}
%\usepackage{hyperref}
% Da bi se promjenio stil citiranja umjesto:
% [authoryear, round]
% staviti:
% [numbers, square]
\usepackage[authoryear, round]{natbib}
\usepackage{amsmath}
\usepackage{subfig}
\usepackage{fixltx2e}
\usepackage{url}
\usepackage{textcomp}
\usepackage{float}

\newcommand{\engl}[1]{(engl.~\emph{#1})}

\begin{document}

\title{Automatizirano prepoznavanje autora teksta\\Projekt iz Strojnog učenja}
\author{Igor Belša \and Tomislav Reicher \and Ivan Krišto}

\maketitle

\thispagestyle{empty}

\tableofcontents
\newpage

\section{Uvod}
Problem automatiziranog prepoznavanja autora teksta \engl{Authorship Attribution}
može se promatrati kao problem klasifikacije teksta na temelju njegovih
lingvističkih značajki. Problemi slični prepoznavanju autora su prepoznavanje
dobi autora, regije iz koje autor potječe ili spola \citep{luyckx2005shallow}.

\citet{kukushkina2001using} navode da se problem prepoznavanja autora teksta
obrađuje još od 1915.~godine u radu ``\emph{Izv.~otd.~russkogo jazyka i
slovesnosti, Imp.~akad.~nauk.}'', Morozov, N., A.~kojem se već 1916.~pridružio
utemeljitelj teorije Markovljevih lanaca, Markov, A., A.,~radom ``\emph{On some
application of statistical method}''.

Prepoznavanje autora može pomoći pri indeksiranju dokumenata, filtriranju ili
hijerarhiskoj kategorizaciji web stranica i web pretraživača
\citep{luyckx2005shallow}. Bitno je napomenuti da prepoznavanje autora i
njegovih karakteristika se razlikuje od detekcije plagijata. Detekcija
plagijata pokušava odrediti sličnost između dva u suštini različita
djela, ali nije u mogućnosti otkriti jesu li proizvod istog autora
\citep{de2001mining}.

Problem prepoznavanja autora može se podijeliti u tri kategorije
\citep{zhao2005effective}: binarnu, višerazrednu i jednorazrednu klasifikaciju.
Binarna klasifikacija rješava problem kad je za skup tekstova poznato da
pripadaju jednom od dva autora. Višerazredna klasifikacija je proširenje
binarne na više autora. Jednorazredna se primjenjuje u situacijama kad je
dio tekstova napisan od poznatog autora, a autori ostalih tekstova su
nepoznati. Ta klasifikacija pokušava odgovoriti na pitanje pripada li neki
tekst poznatom autoru ili ne.

U ovom radu opisan je problem višerazredne klasifikacije za tekstove na
Hrvatskom jeziku.

U 2.~odjeljku opisane su metode koje su u drugim radovima korištene za
problem prepoznavanja autora i srodne probleme. Odjeljak 3 daje kratki pregled
sustava, način na koji su komponente u sustavu povezane i na koji način se
sustav može koristiti. Slijedi 4.~odjeljak s opisom korištenih podataka. Možda
za rad najbitniji odjeljak je 5. U njemu je dat opis korištenih metoda za
izvlačenje značajki iz tekstova, popis svih korištenih značajki i njihove
specifičnosti. Na 5.~odjeljak usko se veže 6.~sa evaluacijom navedenih metoda.
Implementacija je opisana u 7.~odjeljku, a zaključak je dat u 8.
 
\section{Srodna rješenja}
% TODO: Napisati, ovdje ima puno.
\emph{Iskoristiti ono sa google docsa.}

\section{Model sustava}
Namjena razvijenog sustava je prepoznati autora sustavu dosad neviđenog teksta.
Autor se može prepoznati samo ako je sustav naučen nad skupom tekstova koji je
sadržavao i tekstove tog autora. Sustav ima dva načina korištenja, učenje i
određivanje autora. Da bi se sustav mogao koristiti za određivanje autora, prvo
mora biti naučen. Apstraktni model je dat slikom \ref{fig:model-sustava}, a
detalji o načinu korištenja se nalaze u \ref{sec:implementacija}.~odjeljku.

\begin{figure}[htb]
\begin{center}		
\begin{picture}(470,90)		
\put(5,70){\makebox(40,15){\texttt{XML}}}
\put(38,77){\vector(1,0){15}}
\put(55,70){\framebox(90,15){\textit{Izlučivanje značajki}}}		
\put(145,77){\vector(1,0){15}}
\put(160,70){\framebox(90,15){\textit{Učenje}}}		
\put(250,77){\vector(1,0){15}}
\put(285,70){\makebox(40,15){\texttt{prepoznavatelj}}}

\put(1,30){\makebox(40,15){\texttt{tekst}}}
\put(38,37){\vector(1,0){15}}
\put(55,30){\framebox(90,15){\textit{Izlučivanje značajki}}}		
\put(145,37){\vector(1,0){15}}
\put(160,30){\framebox(90,15){\textit{Prepoznavanje}}}		
\put(250,37){\vector(1,0){15}}
\put(205,15){\vector(0,1){15}}
\put(165,0){\makebox(80,15){\texttt{naučeni prepoznavatelj}}}
\put(262,30){\makebox(40,15){\texttt{autor}}}
\end{picture}		
\caption{Model sustava}		
\label{fig:model-sustava}
\end{center}		
\end{figure}

\section{Korišteni podatci}
\label{sec:podatci}
Pri izradi sustava korištena je mrežna arhiva kolumni ``Jutarnjeg lista'' do
14.~studenog, 2009.~-- ``\emph{Komentari i kolumne na aktualna događanja u
Hrvatskoj - Jutarnji.hr}''.\footnote{\url{http://www.jutarnji.hr/komentari/}}

Arhiva je spremljena kao XML\footnote{Extensible Markup Language --
\url{http://www.w3.org/XML/}} dokument po KTLab\footnote{Knowledge Technologies
Lab, FER -- \url{http://ktlab.fer.hr}} \texttt{documentSet} shemi dostupnoj na
\url{http://ktlab.fer.hr/download/documentSet.xsd}. Budući da se veliki broj
članaka u mrežnoj arhivi našao u više kopija, one su pri izradi XML arhive
preskočene.

U arhivi se nalaze tekstovi 25 autora sa ukupno 4360 članaka koji zajedno broje
3400813 riječi. Detaljne statistike arhive nalaze se na slikama
\ref{fig:articlesPerAuthor}\footnote{Broj članaka za autora na mjestu 21 nije
greška. Radi se o Živku Kustiću koji je imao ukupno 1285 objavljenih kolumni do
datuma kada je napravljena arhiva.}, \ref{fig:wordsPerAuthor} i
\ref{fig:avgWordsPerAuthorArticle}.

\begin{figure}[!h]
\begin{center}
% GNUPLOT: LaTeX picture
\setlength{\unitlength}{0.240900pt}
\ifx\plotpoint\undefined\newsavebox{\plotpoint}\fi
\sbox{\plotpoint}{\rule[-0.200pt]{0.400pt}{0.400pt}}%
\begin{picture}(1500,900)(0,0)
\sbox{\plotpoint}{\rule[-0.200pt]{0.400pt}{0.400pt}}%
\put(260.0,131.0){\rule[-0.200pt]{4.818pt}{0.400pt}}
\put(240,131){\makebox(0,0)[r]{$10^{1}$}}
\put(1439.0,131.0){\rule[-0.200pt]{4.818pt}{0.400pt}}
\put(260.0,204.0){\rule[-0.200pt]{2.409pt}{0.400pt}}
\put(1449.0,204.0){\rule[-0.200pt]{2.409pt}{0.400pt}}
\put(260.0,247.0){\rule[-0.200pt]{2.409pt}{0.400pt}}
\put(1449.0,247.0){\rule[-0.200pt]{2.409pt}{0.400pt}}
\put(260.0,277.0){\rule[-0.200pt]{2.409pt}{0.400pt}}
\put(1449.0,277.0){\rule[-0.200pt]{2.409pt}{0.400pt}}
\put(260.0,301.0){\rule[-0.200pt]{2.409pt}{0.400pt}}
\put(1449.0,301.0){\rule[-0.200pt]{2.409pt}{0.400pt}}
\put(260.0,320.0){\rule[-0.200pt]{2.409pt}{0.400pt}}
\put(1449.0,320.0){\rule[-0.200pt]{2.409pt}{0.400pt}}
\put(260.0,336.0){\rule[-0.200pt]{2.409pt}{0.400pt}}
\put(1449.0,336.0){\rule[-0.200pt]{2.409pt}{0.400pt}}
\put(260.0,350.0){\rule[-0.200pt]{2.409pt}{0.400pt}}
\put(1449.0,350.0){\rule[-0.200pt]{2.409pt}{0.400pt}}
\put(260.0,363.0){\rule[-0.200pt]{2.409pt}{0.400pt}}
\put(1449.0,363.0){\rule[-0.200pt]{2.409pt}{0.400pt}}
\put(260.0,374.0){\rule[-0.200pt]{4.818pt}{0.400pt}}
\put(240,374){\makebox(0,0)[r]{$10^{2}$}}
\put(1439.0,374.0){\rule[-0.200pt]{4.818pt}{0.400pt}}
\put(260.0,447.0){\rule[-0.200pt]{2.409pt}{0.400pt}}
\put(1449.0,447.0){\rule[-0.200pt]{2.409pt}{0.400pt}}
\put(260.0,490.0){\rule[-0.200pt]{2.409pt}{0.400pt}}
\put(1449.0,490.0){\rule[-0.200pt]{2.409pt}{0.400pt}}
\put(260.0,520.0){\rule[-0.200pt]{2.409pt}{0.400pt}}
\put(1449.0,520.0){\rule[-0.200pt]{2.409pt}{0.400pt}}
\put(260.0,544.0){\rule[-0.200pt]{2.409pt}{0.400pt}}
\put(1449.0,544.0){\rule[-0.200pt]{2.409pt}{0.400pt}}
\put(260.0,563.0){\rule[-0.200pt]{2.409pt}{0.400pt}}
\put(1449.0,563.0){\rule[-0.200pt]{2.409pt}{0.400pt}}
\put(260.0,579.0){\rule[-0.200pt]{2.409pt}{0.400pt}}
\put(1449.0,579.0){\rule[-0.200pt]{2.409pt}{0.400pt}}
\put(260.0,593.0){\rule[-0.200pt]{2.409pt}{0.400pt}}
\put(1449.0,593.0){\rule[-0.200pt]{2.409pt}{0.400pt}}
\put(260.0,606.0){\rule[-0.200pt]{2.409pt}{0.400pt}}
\put(1449.0,606.0){\rule[-0.200pt]{2.409pt}{0.400pt}}
\put(260.0,617.0){\rule[-0.200pt]{4.818pt}{0.400pt}}
\put(240,617){\makebox(0,0)[r]{$10^{3}$}}
\put(1439.0,617.0){\rule[-0.200pt]{4.818pt}{0.400pt}}
\put(260.0,690.0){\rule[-0.200pt]{2.409pt}{0.400pt}}
\put(1449.0,690.0){\rule[-0.200pt]{2.409pt}{0.400pt}}
\put(260.0,733.0){\rule[-0.200pt]{2.409pt}{0.400pt}}
\put(1449.0,733.0){\rule[-0.200pt]{2.409pt}{0.400pt}}
\put(260.0,763.0){\rule[-0.200pt]{2.409pt}{0.400pt}}
\put(1449.0,763.0){\rule[-0.200pt]{2.409pt}{0.400pt}}
\put(260.0,787.0){\rule[-0.200pt]{2.409pt}{0.400pt}}
\put(1449.0,787.0){\rule[-0.200pt]{2.409pt}{0.400pt}}
\put(260.0,806.0){\rule[-0.200pt]{2.409pt}{0.400pt}}
\put(1449.0,806.0){\rule[-0.200pt]{2.409pt}{0.400pt}}
\put(260.0,822.0){\rule[-0.200pt]{2.409pt}{0.400pt}}
\put(1449.0,822.0){\rule[-0.200pt]{2.409pt}{0.400pt}}
\put(260.0,836.0){\rule[-0.200pt]{2.409pt}{0.400pt}}
\put(1449.0,836.0){\rule[-0.200pt]{2.409pt}{0.400pt}}
\put(260.0,849.0){\rule[-0.200pt]{2.409pt}{0.400pt}}
\put(1449.0,849.0){\rule[-0.200pt]{2.409pt}{0.400pt}}
\put(260.0,860.0){\rule[-0.200pt]{4.818pt}{0.400pt}}
\put(240,860){\makebox(0,0)[r]{$10^{4}$}}
\put(1439.0,860.0){\rule[-0.200pt]{4.818pt}{0.400pt}}
\put(260.0,131.0){\rule[-0.200pt]{0.400pt}{4.818pt}}
\put(260,90){\makebox(0,0){ 0}}
\put(260.0,840.0){\rule[-0.200pt]{0.400pt}{4.818pt}}
\put(491.0,131.0){\rule[-0.200pt]{0.400pt}{4.818pt}}
\put(491,90){\makebox(0,0){ 5}}
\put(491.0,840.0){\rule[-0.200pt]{0.400pt}{4.818pt}}
\put(721.0,131.0){\rule[-0.200pt]{0.400pt}{4.818pt}}
\put(721,90){\makebox(0,0){ 10}}
\put(721.0,840.0){\rule[-0.200pt]{0.400pt}{4.818pt}}
\put(952.0,131.0){\rule[-0.200pt]{0.400pt}{4.818pt}}
\put(952,90){\makebox(0,0){ 15}}
\put(952.0,840.0){\rule[-0.200pt]{0.400pt}{4.818pt}}
\put(1182.0,131.0){\rule[-0.200pt]{0.400pt}{4.818pt}}
\put(1182,90){\makebox(0,0){ 20}}
\put(1182.0,840.0){\rule[-0.200pt]{0.400pt}{4.818pt}}
\put(1413.0,131.0){\rule[-0.200pt]{0.400pt}{4.818pt}}
\put(1413,90){\makebox(0,0){ 25}}
\put(1413.0,840.0){\rule[-0.200pt]{0.400pt}{4.818pt}}
\put(260.0,131.0){\rule[-0.200pt]{0.400pt}{175.616pt}}
\put(260.0,131.0){\rule[-0.200pt]{288.839pt}{0.400pt}}
\put(1459.0,131.0){\rule[-0.200pt]{0.400pt}{175.616pt}}
\put(260.0,860.0){\rule[-0.200pt]{288.839pt}{0.400pt}}
\put(859,29){\makebox(0,0){Author}}
\put(306.0,131.0){\rule[-0.200pt]{0.400pt}{93.951pt}}
\put(352.0,131.0){\rule[-0.200pt]{0.400pt}{69.138pt}}
\put(398.0,131.0){\rule[-0.200pt]{0.400pt}{58.057pt}}
\put(444.0,131.0){\rule[-0.200pt]{0.400pt}{47.698pt}}
\put(491.0,131.0){\rule[-0.200pt]{0.400pt}{36.376pt}}
\put(537.0,131.0){\rule[-0.200pt]{0.400pt}{89.374pt}}
\put(583.0,131.0){\rule[-0.200pt]{0.400pt}{30.353pt}}
\put(629.0,131.0){\rule[-0.200pt]{0.400pt}{63.598pt}}
\put(675.0,131.0){\rule[-0.200pt]{0.400pt}{61.911pt}}
\put(721.0,131.0){\rule[-0.200pt]{0.400pt}{21.199pt}}
\put(767.0,131.0){\rule[-0.200pt]{0.400pt}{33.244pt}}
\put(813.0,131.0){\rule[-0.200pt]{0.400pt}{38.303pt}}
\put(859.0,131.0){\rule[-0.200pt]{0.400pt}{37.580pt}}
\put(906.0,131.0){\rule[-0.200pt]{0.400pt}{45.530pt}}
\put(952.0,131.0){\rule[-0.200pt]{0.400pt}{70.102pt}}
\put(998.0,131.0){\rule[-0.200pt]{0.400pt}{54.202pt}}
\put(1044.0,131.0){\rule[-0.200pt]{0.400pt}{64.561pt}}
\put(1090.0,131.0){\rule[-0.200pt]{0.400pt}{64.802pt}}
\put(1136.0,131.0){\rule[-0.200pt]{0.400pt}{8.672pt}}
\put(1182.0,131.0){\rule[-0.200pt]{0.400pt}{82.147pt}}
\put(1228.0,131.0){\rule[-0.200pt]{0.400pt}{124.786pt}}
\put(1275.0,131.0){\rule[-0.200pt]{0.400pt}{72.511pt}}
\put(1321.0,131.0){\rule[-0.200pt]{0.400pt}{77.329pt}}
\put(1367.0,131.0){\rule[-0.200pt]{0.400pt}{71.065pt}}
\put(1413.0,131.0){\rule[-0.200pt]{0.400pt}{88.892pt}}
\put(306,521){\circle{12}}
\put(352,418){\circle{12}}
\put(398,372){\circle{12}}
\put(444,329){\circle{12}}
\put(491,282){\circle{12}}
\put(537,502){\circle{12}}
\put(583,257){\circle{12}}
\put(629,395){\circle{12}}
\put(675,388){\circle{12}}
\put(721,219){\circle{12}}
\put(767,269){\circle{12}}
\put(813,290){\circle{12}}
\put(859,287){\circle{12}}
\put(906,320){\circle{12}}
\put(952,422){\circle{12}}
\put(998,356){\circle{12}}
\put(1044,399){\circle{12}}
\put(1090,400){\circle{12}}
\put(1136,167){\circle{12}}
\put(1182,472){\circle{12}}
\put(1228,649){\circle{12}}
\put(1275,432){\circle{12}}
\put(1321,452){\circle{12}}
\put(1367,426){\circle{12}}
\put(1413,500){\circle{12}}
\put(260.0,131.0){\rule[-0.200pt]{0.400pt}{175.616pt}}
\put(260.0,131.0){\rule[-0.200pt]{288.839pt}{0.400pt}}
\put(1459.0,131.0){\rule[-0.200pt]{0.400pt}{175.616pt}}
\put(260.0,860.0){\rule[-0.200pt]{288.839pt}{0.400pt}}
\end{picture}

\end{center}
\caption{Broj članaka po autoru.}
\label{fig:articlesPerAuthor}
\end{figure}

\begin{figure}[!h]
\begin{center}
% GNUPLOT: LaTeX picture
\setlength{\unitlength}{0.240900pt}
\ifx\plotpoint\undefined\newsavebox{\plotpoint}\fi
\sbox{\plotpoint}{\rule[-0.200pt]{0.400pt}{0.400pt}}%
\begin{picture}(1500,900)(0,0)
\sbox{\plotpoint}{\rule[-0.200pt]{0.400pt}{0.400pt}}%
\put(240.0,131.0){\rule[-0.200pt]{4.818pt}{0.400pt}}
\put(220,131){\makebox(0,0)[r]{ 0}}
\put(1439.0,131.0){\rule[-0.200pt]{4.818pt}{0.400pt}}
\put(240.0,252.0){\rule[-0.200pt]{4.818pt}{0.400pt}}
\put(220,252){\makebox(0,0)[r]{ 100000}}
\put(1439.0,252.0){\rule[-0.200pt]{4.818pt}{0.400pt}}
\put(240.0,374.0){\rule[-0.200pt]{4.818pt}{0.400pt}}
\put(220,374){\makebox(0,0)[r]{ 200000}}
\put(1439.0,374.0){\rule[-0.200pt]{4.818pt}{0.400pt}}
\put(240.0,495.0){\rule[-0.200pt]{4.818pt}{0.400pt}}
\put(220,495){\makebox(0,0)[r]{ 300000}}
\put(1439.0,495.0){\rule[-0.200pt]{4.818pt}{0.400pt}}
\put(240.0,617.0){\rule[-0.200pt]{4.818pt}{0.400pt}}
\put(220,617){\makebox(0,0)[r]{ 400000}}
\put(1439.0,617.0){\rule[-0.200pt]{4.818pt}{0.400pt}}
\put(240.0,738.0){\rule[-0.200pt]{4.818pt}{0.400pt}}
\put(220,738){\makebox(0,0)[r]{ 500000}}
\put(1439.0,738.0){\rule[-0.200pt]{4.818pt}{0.400pt}}
\put(240.0,860.0){\rule[-0.200pt]{4.818pt}{0.400pt}}
\put(220,860){\makebox(0,0)[r]{ 600000}}
\put(1439.0,860.0){\rule[-0.200pt]{4.818pt}{0.400pt}}
\put(240.0,131.0){\rule[-0.200pt]{0.400pt}{4.818pt}}
\put(240,90){\makebox(0,0){ 0}}
\put(240.0,840.0){\rule[-0.200pt]{0.400pt}{4.818pt}}
\put(474.0,131.0){\rule[-0.200pt]{0.400pt}{4.818pt}}
\put(474,90){\makebox(0,0){ 5}}
\put(474.0,840.0){\rule[-0.200pt]{0.400pt}{4.818pt}}
\put(709.0,131.0){\rule[-0.200pt]{0.400pt}{4.818pt}}
\put(709,90){\makebox(0,0){ 10}}
\put(709.0,840.0){\rule[-0.200pt]{0.400pt}{4.818pt}}
\put(943.0,131.0){\rule[-0.200pt]{0.400pt}{4.818pt}}
\put(943,90){\makebox(0,0){ 15}}
\put(943.0,840.0){\rule[-0.200pt]{0.400pt}{4.818pt}}
\put(1178.0,131.0){\rule[-0.200pt]{0.400pt}{4.818pt}}
\put(1178,90){\makebox(0,0){ 20}}
\put(1178.0,840.0){\rule[-0.200pt]{0.400pt}{4.818pt}}
\put(1412.0,131.0){\rule[-0.200pt]{0.400pt}{4.818pt}}
\put(1412,90){\makebox(0,0){ 25}}
\put(1412.0,840.0){\rule[-0.200pt]{0.400pt}{4.818pt}}
\put(240.0,131.0){\rule[-0.200pt]{0.400pt}{175.616pt}}
\put(240.0,131.0){\rule[-0.200pt]{293.657pt}{0.400pt}}
\put(1459.0,131.0){\rule[-0.200pt]{0.400pt}{175.616pt}}
\put(240.0,860.0){\rule[-0.200pt]{293.657pt}{0.400pt}}
\put(849,29){\makebox(0,0){Autor}}
\put(287.0,131.0){\rule[-0.200pt]{0.400pt}{157.549pt}}
\put(334.0,131.0){\rule[-0.200pt]{0.400pt}{64.320pt}}
\put(381.0,131.0){\rule[-0.200pt]{0.400pt}{13.731pt}}
\put(428.0,131.0){\rule[-0.200pt]{0.400pt}{9.877pt}}
\put(474.0,131.0){\rule[-0.200pt]{0.400pt}{14.213pt}}
\put(521.0,131.0){\rule[-0.200pt]{0.400pt}{45.530pt}}
\put(568.0,131.0){\rule[-0.200pt]{0.400pt}{6.022pt}}
\put(615.0,131.0){\rule[-0.200pt]{0.400pt}{33.967pt}}
\put(662.0,131.0){\rule[-0.200pt]{0.400pt}{29.390pt}}
\put(709.0,131.0){\rule[-0.200pt]{0.400pt}{5.541pt}}
\put(756.0,131.0){\rule[-0.200pt]{0.400pt}{6.022pt}}
\put(803.0,131.0){\rule[-0.200pt]{0.400pt}{7.227pt}}
\put(850.0,131.0){\rule[-0.200pt]{0.400pt}{4.577pt}}
\put(896.0,131.0){\rule[-0.200pt]{0.400pt}{9.154pt}}
\put(943.0,131.0){\rule[-0.200pt]{0.400pt}{31.558pt}}
\put(990.0,131.0){\rule[-0.200pt]{0.400pt}{22.885pt}}
\put(1037.0,131.0){\rule[-0.200pt]{0.400pt}{32.040pt}}
\put(1084.0,131.0){\rule[-0.200pt]{0.400pt}{14.936pt}}
\put(1131.0,131.0){\rule[-0.200pt]{0.400pt}{3.854pt}}
\put(1178.0,131.0){\rule[-0.200pt]{0.400pt}{82.147pt}}
\put(1225.0,131.0){\rule[-0.200pt]{0.400pt}{136.349pt}}
\put(1271.0,131.0){\rule[-0.200pt]{0.400pt}{40.712pt}}
\put(1318.0,131.0){\rule[-0.200pt]{0.400pt}{67.934pt}}
\put(1365.0,131.0){\rule[-0.200pt]{0.400pt}{23.126pt}}
\put(1412.0,131.0){\rule[-0.200pt]{0.400pt}{132.254pt}}
\put(287,785){\circle{12}}
\put(334,398){\circle{12}}
\put(381,188){\circle{12}}
\put(428,172){\circle{12}}
\put(474,190){\circle{12}}
\put(521,320){\circle{12}}
\put(568,156){\circle{12}}
\put(615,272){\circle{12}}
\put(662,253){\circle{12}}
\put(709,154){\circle{12}}
\put(756,156){\circle{12}}
\put(803,161){\circle{12}}
\put(850,150){\circle{12}}
\put(896,169){\circle{12}}
\put(943,262){\circle{12}}
\put(990,226){\circle{12}}
\put(1037,264){\circle{12}}
\put(1084,193){\circle{12}}
\put(1131,147){\circle{12}}
\put(1178,472){\circle{12}}
\put(1225,697){\circle{12}}
\put(1271,300){\circle{12}}
\put(1318,413){\circle{12}}
\put(1365,227){\circle{12}}
\put(1412,680){\circle{12}}
\put(240.0,131.0){\rule[-0.200pt]{0.400pt}{175.616pt}}
\put(240.0,131.0){\rule[-0.200pt]{293.657pt}{0.400pt}}
\put(1459.0,131.0){\rule[-0.200pt]{0.400pt}{175.616pt}}
\put(240.0,860.0){\rule[-0.200pt]{293.657pt}{0.400pt}}
\end{picture}

\end{center}
\caption{Broj riječi po autoru.}
\label{fig:wordsPerAuthor}
\end{figure}

\begin{figure}[!h]
\begin{center}
% GNUPLOT: LaTeX picture
\setlength{\unitlength}{0.240900pt}
\ifx\plotpoint\undefined\newsavebox{\plotpoint}\fi
\sbox{\plotpoint}{\rule[-0.200pt]{0.400pt}{0.400pt}}%
\begin{picture}(1500,900)(0,0)
\sbox{\plotpoint}{\rule[-0.200pt]{0.400pt}{0.400pt}}%
\put(200.0,131.0){\rule[-0.200pt]{4.818pt}{0.400pt}}
\put(180,131){\makebox(0,0)[r]{ 200}}
\put(1439.0,131.0){\rule[-0.200pt]{4.818pt}{0.400pt}}
\put(200.0,253.0){\rule[-0.200pt]{4.818pt}{0.400pt}}
\put(180,253){\makebox(0,0)[r]{ 400}}
\put(1439.0,253.0){\rule[-0.200pt]{4.818pt}{0.400pt}}
\put(200.0,374.0){\rule[-0.200pt]{4.818pt}{0.400pt}}
\put(180,374){\makebox(0,0)[r]{ 600}}
\put(1439.0,374.0){\rule[-0.200pt]{4.818pt}{0.400pt}}
\put(200.0,496.0){\rule[-0.200pt]{4.818pt}{0.400pt}}
\put(180,496){\makebox(0,0)[r]{ 800}}
\put(1439.0,496.0){\rule[-0.200pt]{4.818pt}{0.400pt}}
\put(200.0,617.0){\rule[-0.200pt]{4.818pt}{0.400pt}}
\put(180,617){\makebox(0,0)[r]{ 1000}}
\put(1439.0,617.0){\rule[-0.200pt]{4.818pt}{0.400pt}}
\put(200.0,739.0){\rule[-0.200pt]{4.818pt}{0.400pt}}
\put(180,739){\makebox(0,0)[r]{ 1200}}
\put(1439.0,739.0){\rule[-0.200pt]{4.818pt}{0.400pt}}
\put(200.0,860.0){\rule[-0.200pt]{4.818pt}{0.400pt}}
\put(180,860){\makebox(0,0)[r]{ 1400}}
\put(1439.0,860.0){\rule[-0.200pt]{4.818pt}{0.400pt}}
\put(200.0,131.0){\rule[-0.200pt]{0.400pt}{4.818pt}}
\put(200,90){\makebox(0,0){ 0}}
\put(200.0,840.0){\rule[-0.200pt]{0.400pt}{4.818pt}}
\put(442.0,131.0){\rule[-0.200pt]{0.400pt}{4.818pt}}
\put(442,90){\makebox(0,0){ 5}}
\put(442.0,840.0){\rule[-0.200pt]{0.400pt}{4.818pt}}
\put(684.0,131.0){\rule[-0.200pt]{0.400pt}{4.818pt}}
\put(684,90){\makebox(0,0){ 10}}
\put(684.0,840.0){\rule[-0.200pt]{0.400pt}{4.818pt}}
\put(926.0,131.0){\rule[-0.200pt]{0.400pt}{4.818pt}}
\put(926,90){\makebox(0,0){ 15}}
\put(926.0,840.0){\rule[-0.200pt]{0.400pt}{4.818pt}}
\put(1168.0,131.0){\rule[-0.200pt]{0.400pt}{4.818pt}}
\put(1168,90){\makebox(0,0){ 20}}
\put(1168.0,840.0){\rule[-0.200pt]{0.400pt}{4.818pt}}
\put(1411.0,131.0){\rule[-0.200pt]{0.400pt}{4.818pt}}
\put(1411,90){\makebox(0,0){ 25}}
\put(1411.0,840.0){\rule[-0.200pt]{0.400pt}{4.818pt}}
\put(200.0,131.0){\rule[-0.200pt]{0.400pt}{175.616pt}}
\put(200.0,131.0){\rule[-0.200pt]{303.293pt}{0.400pt}}
\put(1459.0,131.0){\rule[-0.200pt]{0.400pt}{175.616pt}}
\put(200.0,860.0){\rule[-0.200pt]{303.293pt}{0.400pt}}
\put(829,29){\makebox(0,0){Autor}}
\put(248.0,131.0){\rule[-0.200pt]{0.400pt}{147.913pt}}
\put(297.0,131.0){\rule[-0.200pt]{0.400pt}{170.557pt}}
\put(345.0,131.0){\rule[-0.200pt]{0.400pt}{32.762pt}}
\put(394.0,131.0){\rule[-0.200pt]{0.400pt}{38.544pt}}
\put(442.0,131.0){\rule[-0.200pt]{0.400pt}{120.209pt}}
\put(491.0,131.0){\rule[-0.200pt]{0.400pt}{40.953pt}}
\put(539.0,131.0){\rule[-0.200pt]{0.400pt}{52.275pt}}
\put(587.0,131.0){\rule[-0.200pt]{0.400pt}{91.542pt}}
\put(636.0,131.0){\rule[-0.200pt]{0.400pt}{85.760pt}}
\put(684.0,131.0){\rule[-0.200pt]{0.400pt}{79.979pt}}
\put(733.0,131.0){\rule[-0.200pt]{0.400pt}{43.362pt}}
\put(781.0,131.0){\rule[-0.200pt]{0.400pt}{42.157pt}}
\put(829.0,131.0){\rule[-0.200pt]{0.400pt}{17.827pt}}
\put(878.0,131.0){\rule[-0.200pt]{0.400pt}{37.099pt}}
\put(926.0,131.0){\rule[-0.200pt]{0.400pt}{75.643pt}}
\put(975.0,131.0){\rule[-0.200pt]{0.400pt}{91.060pt}}
\put(1023.0,131.0){\rule[-0.200pt]{0.400pt}{87.928pt}}
\put(1072.0,131.0){\rule[-0.200pt]{0.400pt}{22.645pt}}
\put(1120.0,131.0){\rule[-0.200pt]{0.400pt}{86.001pt}}
\put(1168.0,131.0){\rule[-0.200pt]{0.400pt}{121.895pt}}
\put(1217.0,131.0){\rule[-0.200pt]{0.400pt}{17.104pt}}
\put(1265.0,131.0){\rule[-0.200pt]{0.400pt}{79.979pt}}
\put(1314.0,131.0){\rule[-0.200pt]{0.400pt}{122.859pt}}
\put(1362.0,131.0){\rule[-0.200pt]{0.400pt}{34.208pt}}
\put(1411.0,131.0){\rule[-0.200pt]{0.400pt}{150.803pt}}
\put(248,745){\circle{12}}
\put(297,839){\circle{12}}
\put(345,267){\circle{12}}
\put(394,291){\circle{12}}
\put(442,630){\circle{12}}
\put(491,301){\circle{12}}
\put(539,348){\circle{12}}
\put(587,511){\circle{12}}
\put(636,487){\circle{12}}
\put(684,463){\circle{12}}
\put(733,311){\circle{12}}
\put(781,306){\circle{12}}
\put(829,205){\circle{12}}
\put(878,285){\circle{12}}
\put(926,445){\circle{12}}
\put(975,509){\circle{12}}
\put(1023,496){\circle{12}}
\put(1072,225){\circle{12}}
\put(1120,488){\circle{12}}
\put(1168,637){\circle{12}}
\put(1217,202){\circle{12}}
\put(1265,463){\circle{12}}
\put(1314,641){\circle{12}}
\put(1362,273){\circle{12}}
\put(1411,757){\circle{12}}
\put(200.0,131.0){\rule[-0.200pt]{0.400pt}{175.616pt}}
\put(200.0,131.0){\rule[-0.200pt]{303.293pt}{0.400pt}}
\put(1459.0,131.0){\rule[-0.200pt]{0.400pt}{175.616pt}}
\put(200.0,860.0){\rule[-0.200pt]{303.293pt}{0.400pt}}
\end{picture}

\end{center}
\caption{Prosječna duljina članka po autoru.}
\label{fig:avgWordsPerAuthorArticle}
\end{figure}

%\subsection{Preprocesiranje}
% Ako ovo uopće budemo imali, jer zasad imamo samo izvlačenje značajki

\section{Izvlačenje značajki iz dokumenata}
% Opisati problem i par riječi ukratko o svemu

\subsection{Metoda1}
\subsection{Metoda2}
\subsection{Metoda3}

\section{Evaluacija}

\section{Implementacija}
\label{sec:implementacija}
Implementacija je napisana u programskim jezicima Javi (glavni sustav) te
Pythonu (skripta \texttt{JutarnjiKolumneArhiver.py}). Navedena skripta služi za
dohvat arhive tekstova sa ``\emph{Komentari i kolumne na aktualna događanja u Hrvatskoj -
Jutarnji.hr}''.\footnote{\url{http://www.jutarnji.hr/komentari/}}

% TODO: UCD Dohvat članaka i izlučivanje značajki
% TODO: UCD Učenje sustava
% TODO: UCD Prepoznavanje autora.

Za klasifikaciju iskorištena je biblioteka \emph{libsvm} \citep{CC01a}. Veza
\emph{libsvm} biblioteke i sustava izvedena je \texttt{LibsvmRecognizer}
razredom koji istovremeno implementira sučelja \texttt{RecognizerTrainer} i
\texttt{AuthorRecognizer}.

Korisnik sustavu može pristupiti iz komandne linije preko \texttt{CLITrainer}
(učenje sustava) i \texttt{CLIRecognizer} (prepoznavatelj autora) ili preko
grafičkog sučelja iz \texttt{GUI} razreda. Svi navedeni razredi nalaze se u
\texttt{hr.fer.zemris.aa.main} paketu. Primjer pokretanja sustava iz
komandne linije:
\begin{verbatim}
  java hr.fer.zemris.aa.main.CLITrainer \
  skup_za_ucenje-skup_aa.xml \
  skup_aa.model
  
  java hr.fer.zemris.aa.main.CLIRecognizer \
  tekst_nepoznatog_autora.txt \
  skup_aa.model
\end{verbatim}


\section{Zaključak}

\bibliography{literatura}
\bibliographystyle{plainnat}

\newpage
\appendix
\section{Doprinos svakog člana tima}
\begin{description}
\item[Ivan Krišto:]
\item[Igor Belša:]
\item[Tomislav Reicher:]
\end{description}

\section{Raspodjela bodova}
\begin{description}
\item[Ivan Krišto:] p1\%
\item[Igor Belša:] p2\%
\item[Tomislav Reicher:] p3\%
\end{description}

\end{document}
